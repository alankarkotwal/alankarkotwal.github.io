\documentclass[11pt]{article}

\usepackage[margin=0.5in]{geometry}
\pagenumbering{gobble}

\begin{document}
\noindent Hi. I'm Alankar, and what I'm going to talk about today is geometry and physics, and how the combination of those two can help us in the seemingly \textit{impossible} task of distance measurement in Astronomy. You see, the problem with Astrophysics is the following: you're a physicist, you design an experiment, perform it, check if your hypothesis is good, and be happy. You're an astrophysicist, you don't have that luxury. You have to wait for the universe to do its experiments, and understand what the universe does, which makes it so much more difficult. The fact that we've measured almost all distances that are to be measured is a huge achievement, and by that scale me putting even `Earth' as where I am from would be inappropriate.

\noindent\rule[0.5ex]{\linewidth}{1pt}

\noindent To get an idea about the scales we're about to talk about, consider the smallest scale we'll be talking about. The Sun is about 150 million kilometers from us. If you divide that distance by the length of the Infinite Corridor, which is about half a kilometer, you'll find out that you'll need to complete the irritating walk across the corridor about 375 million times to reach the sun. Assuming an average human walks at 5 kilometers an hour, that will take you 3.5 thousand years. No rulers that big can be built! This is a recurring theme, you measure distances by their effects and not by rulers.

\noindent\rule[0.5ex]{\linewidth}{1pt}

\noindent Now what is this `ladder' business all about? It turns out, the pattern is that measurements at smaller distances, of the distances of the order of the Solar System, for example, influence measurement to the nearest stars ($\sim$ 100 million kilometers), $\alpha$ Centauri (41 million million kilometers) for example. This further helps us measure the distance to Betelgeuse (614 million million million kilometers) and to this particular Gamma-ray burster, that is about as far as light would have traveled from the birth of the universe to now.

\noindent\rule[0.5ex]{\linewidth}{1pt}

\noindent So off we go at the scale of the Solar System, where gravity ensures that we can rely on Kepler's laws, in particular the third. We begin by estimating distances to the inner planets. These are the orbits of Venus and Earth, respectively. What's so special about this particular arrangement? It is that the separation between Venus and the Sun cannot make any greater angle than this! Move Venus either way, and the angle decreases. We can observe this maximum angle! Measure the angle between Venus and the Sun in the sky for a long long time, and look at the maximum! Simple trigonometry in this triangle gives us the relation between the distances between the Sun and the Earth (the au), and the Sun and Venus. But how do we measure the absolute scale? Another ingenious trick goes into doing this. The event here is the transit of Venus. This is when Venus passes on the solar disk as seen from the Earth. Observers from two different places will see Venus pass through different points on the solar disk, and this difference can be used to find the absolute distance to the Sun. This movement of Venus when seen from two different places is called the `parallax'. Bouncing radar signals off the Sun, of course, give us distances accurate to centimeters. This takes us out to the size of the solar system, which is about 100 au.

\noindent\rule[0.5ex]{\linewidth}{1pt}

\noindent What happens when we go beyond the limits of the solar system? The Sun's gravitational influence doesn't help us here, but geometry does. What we do is use the same parallax trick, except that instead of using two places on the Earth as the baseline, we use two points in the Earth's orbit. We calculated the Earth-Sun distance, and this helps estimate the distance to the star we're considering. We can measure these `angles' to an accuracy of a hundredth of an arc-second, taking us out to about a hundred parsecs.

\noindent\rule[0.5ex]{\linewidth}{1pt}

\noindent This is probably my favorite slide, because of the ingenious use of trigonometry here. Suppose you have a cluster of stars that are moving in a particular direction. You observe them at two points in time. Since one of the positions is farther than the other, you see that the cluster `shrinks' and joining the corresponding stars in the two positions gives us some kind of `convergence' point. You then know the angle between the convergence point and the current direction of the cluster. This gives you a handle on the ratio of how much it moved in this direction with respect to that. Doppler measurements give you how much the cluster moved in this, and you know the angle between this and this, and therefore you have the distance.

\noindent\rule[0.5ex]{\linewidth}{1pt}

\noindent It is believed more than half the stars in the universe exist with a companion star, making the system a binary system. Consider observing such a system from the Earth. You have one star moving one way and the other moving another way. This results in the famous Doppler effect, where spectrum of a source, in particular features in the spectrum shift in relation to the speed relative to the observer. This, combined with the angular size of the orbit as seen from earth, gives us the distance to the star system.

\noindent\rule[0.5ex]{\linewidth}{1pt}

\noindent All we did till now was consider positions of objects. Why not consider the amount of light? We know light falls off as the inverse of the square of the distance, we can use that to estimate the distance if we know the amount of light it intrinsically emits. How do we do that? We look for something that is distance-independent. It turns out that the ratio of, for instance, yellow light to red light that a star emits is independent of distance to it, because both fall as the square of the distance. This is roughly what astronomers call the color of the star. Just like you'd expect with an iron rod, the color of a star is an indicator of its temperature. Now it turns out that if you plot all the stars in the universe on a graph that shows their temperature versus their intrinsic brightness, you get a roughly horizontal band on which you can project your star and get its intrinsic brightness, and hence its distance.  This takes us out to the scale of our galaxy.

\noindent\rule[0.5ex]{\linewidth}{1pt}

\noindent Physics of objects at scales higher than this is less understood. The pattern, therefore, is to use what are called `standard candles', which seem to exhibit some distance-independent property. Examples are what are called Cepheid variable stars, which is a class of pulsating stars showing a relation between their pulsation period and intrinsic brightness. You measure the period, get the brightness, and therefore get its distance. Cepheids have been used to measure the distances to the immediate vicinity of the cluster of galaxies our galaxy is in.

\noindent\rule[0.5ex]{\linewidth}{1pt}

\noindent This is probably a term that excites a lot of people. Supernovae are especially significant because they can be seen over long distances, as the animation (M82) shows. A supernova is, in its essentials, magnificent stellar death. Supernovae cause the central part of the dying star to collapse into a small object, while blowing the outer layers into shells of gases around the compact object. A quick calculation would involve observing the angular rate of expansion of a supernova and measuring its radial speed towards us, yielding the distance to it. More accurately, a class of supernovae have been shown to have an almost-constant, known intrinsic brightness and hence deducing distances to them is easy. Since supernovae can be seen to such large distances, astronomers are waiting desperately for a supernova in the Milky Way, which is long due, in order to study its physics! The best way to troll an astronomer would be to tell him Betelgeuse, a star in Orion that is a potential candidate, has exploded!

\noindent\rule[0.5ex]{\linewidth}{1pt}

\noindent At greater distances where cosmology applies, we can use the well-known Hubble's law to get an estimate of distances from radial velocity using Doppler shifts. The Hubble constant is found from this region, and we hope it works when extrapolated in this region.

\noindent\rule[0.5ex]{\linewidth}{1pt}

\noindent Analysis of galaxy structure seems to suggest that the variance of the velocities in an elliptical galaxy is related to its intrinsic brightness, giving its distance to us.

\noindent\rule[0.5ex]{\linewidth}{1pt}

\noindent A similar expression for spherical galaxies is the relation between the maximum rotation velocity around the center yielding the intrinsic brightness.

\noindent\rule[0.5ex]{\linewidth}{1pt}

\noindent All this was under the assumption that we get enough `information' out from the source that we consider. However, there exist sources which are so faint that you cannot do a spectrum (there is simply not enough energy to do so), yielding no Doppler information at all. However, we can take `broad' bands and try to `learn' a relation between these broad-band brightness-es and distance. This is called `photometric redshift' and is a good exercise in machine learning. Earlier we used the spectrum as an input to a function calculating Doppler shifts and estimating distances, we hope this set of broad-band brightness-es is a good proxy for that. 

\noindent\rule[0.5ex]{\linewidth}{1pt}

\noindent This shows the first results of regressing using the previous method. The x-axis is the actual redshift, mean-subtracted and the y-axis shows the prediction. This is close to $y=x$, like we would want it to be! This shows the results of classification. Everything in the first slot here is a class 1, and so on. This shows a fairly good result in classifying objects.

\noindent\rule[0.5ex]{\linewidth}{1pt}

\noindent This being said, astronomers are always coming up with new ways of measuring distances! This is an evolving and important field with lots of excitement in the future! Questions?

\end{document}
