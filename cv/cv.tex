\documentclass[margin,line]{res}

\usepackage{hyperref}
\usepackage{amsmath}
\usepackage{textcomp}
\usepackage{color}
\usepackage{fontawesome,marvosym,ifsym}
\oddsidemargin = -.5in
\evensidemargin = -.5in
\topmargin = -0.3in
\textheight = 9.75in
\textwidth=6.15in
\itemsep=0in
\parsep=0in
% if using pdflatex:
%\setlength{\pdfpagewidth}{\paperwidth}
%\setlength{\pdfpageheight}{\paperheight} 

\newenvironment{list1}{
  \begin{list}{\ding{113}}{%
      \setlength{\itemsep}{0in}
      \setlength{\parsep}{0in} \setlength{\parskip}{0in}
      \setlength{\topsep}{0in} \setlength{\partopsep}{0in} 
      \setlength{\leftmargin}{0.17in}}}{\end{list}}
\newenvironment{list2}{
  \begin{list}{$\bullet$}{%
      \setlength{\itemsep}{0in}
      \setlength{\parsep}{0in} \setlength{\parskip}{0in}
      \setlength{\topsep}{0in} \setlength{\partopsep}{0in} 
      \setlength{\leftmargin}{0.18in}}}{\end{list}}

\tolerance=1
\emergencystretch=\maxdimen
\hyphenpenalty=10000
\hbadness=10000

\pagenumbering{gobble}

\begin{document}

%\name{\textbf{\huge{Alankar Kotwal}} \vspace*{.05in} \newline  {\sc Senior Undergraduate}\vspace*{.1in}}
\name{\textbf{\huge{Alankar Kotwal}}\ \ \textbar\ \  \normalfont{\large \Email\ \href{mailto:askotwal@andrew.cmu.edu}{askotwal@andrew.cmu.edu}}\ \textbar\ \ \normalfont{\large \Mundus\ \href{http://alankarkotwal.github.io/}{alankarkotwal.github.io}} \vspace*{.1in}}

\begin{resume}

% \section{\sc Contact Information}
% \vspace{.05in}
% \begin{tabular}{@{}p{2.9in}p{.5in}p{3in}}
% Elliot Dunlap Smith Hall 220 & \multicolumn{1}{r}{\it Phone:}  & +1 (412) 616-5496 \\            
% Carnegie Mellon University &\multicolumn{1}{r}{\it E--Mail:}& \href{mailto:askotwal@andrew.cmu.edu}{\textcolor{blue}{askotwal@andrew.cmu.edu}} \\ 
% \end{tabular}

%\vspace*{-0.13in}

\section{\sc Research Interests}
My research interests lie in developing computational imaging techniques for medical applications. I currently work on developing optical imaging systems that exploit the wave nature of light to extract information from the environment that a traditional camera cannot. The goal of my research is using this information to see deep inside tissue using visible light (details \href{http://seebelowtheskin.rice.edu/}{\textcolor{blue}{here}}). I am also interested in developing computational algorithms to ease or automate the understanding of images thus collected.

\vspace*{-0.1in}

In the future, I would like to explore the clinical aspect of research in medical imaging techniques. 

\vspace*{-0.13in}

\section{\sc Education}
{\bf \href{http://www.cmu.edu/}{\textcolor{blue}{Carnegie Mellon University}}} \hfill {\it August 2017 -- present} \\
\vspace*{-.15in}
\begin{list1}
\item[] Doctor of Philosophy Program, \href{http://www.ri.cmu.edu/}{\textcolor{blue}{The Robotics Institute}}
\end{list1}
\vspace*{-.1in}
{\bf \href{http://www.iitb.ac.in/}{\textcolor{blue}{Indian Institute of Technology Bombay}}} \hfill {\it July 2012 -- June 2017} \\
\vspace*{-.15in}
\begin{list1}
\item[] Dual Degree (Bachelor \& Master of Technology), \href{http://www.ee.iitb.ac.in/}{\textcolor{blue}{Electrical Engineering}} 
\item[] Specialization in \textit{Communication and Signal Processing}. CPI: 9.16/10.00
\item[] Awarded the \textit{undergraduate research award} for an exceptional final year project
%\item[] Specialization: {\em Signal Processing}, {\bf CGPA: 9.18/10.00}, Minor: \textit{Computer Science and Engineering}
\end{list1}

\vspace*{-0.13in}

\section{{\sc Relevant \\Publications}}
\begin{list2}
\item Baid, A., Kotwal, A., Bhalodia, R., Awate, S., {\em Joint desmoking, specularity removal, and denoising of laparoscopy images via graphical models and Bayesian inference}. Proc. of the \href{http://biomedicalimaging.org/2017/}{\textcolor{blue} {14$^\text{th}$ International Symposium on Biomedical Imaging (2017)}}. Paper \href{http://ieeexplore.ieee.org/document/7950623/?reload=true}{\textcolor{blue} {here}}.
\item Kotwal, A.*, Bhalodia, R.*, Awate, S., {\em Joint desmoking and denoising of laparoscopy images}, Proc. of the \href{http://biomedicalimaging.org/2016/}{\textcolor{blue} {13$^\text{th}$ International Symposium on Biomedical Imaging (2016)}}. Paper \href{http://ieeexplore.ieee.org/document/7493446/?reload=true}{\textcolor{blue} {here}}.
\item Shah, D.*, Kotwal, A.* and Rajwade, A. V., {\em Designing constrained projections for compressed sensing: mean errors and anomalies with coherence}, accepted for presentation at the \href{https://2018.ieeeglobalsip.org/}{\textcolor{blue}{6$^\text{th}$ IEEE Global Conference on Signal and Information Processing (2018)}}.
\item Kotwal, A. and Rajwade, A. V., {\em Optimizing matrices for compressed sensing using existing goodness measures: negative results, and an alternative}, \href{https://arxiv.org/abs/1707.03355}{\textcolor{blue}{arXiv:1707.03355 [cs.IT]}}.
\item Kotwal, A., Rajwade, A. V., {\em Optimizing codes for source separation in compressed video recovery and color image demosaicing}, \href{https://arxiv.org/abs/1609.02135}{\textcolor{blue} {arXiv:1609.02135 [cs.CV]}}.
%\item Clarke, J. {\em et al.}, {\em Field Robotics, Astrobiology and Mars Analogue Research on the Arkaroola Mars Robot Challenge}, Proc. of the \href{http://www.nssa.com.au/14asrc/14ASRC-proceedings.zip}{\textcolor{blue} {14$^\text{th}$ Australian Space Research Conference (2014)}}. Paper \href{http://alankarkotwal.github.io/pubs/asrc14.pdf}{\textcolor{blue} {here}}.
\end{list2}

\vspace*{-0.13in}

\section{\sc Relevant Research Projects}

%\vspace*{-0.13in}
%\vspace*{-0.1in}

{\bf Coded coherence imaging for seeing through tissue} \\
{\em Advisor: \href{https://www.cs.cmu.edu/~igkioule}{\textcolor{blue}{Prof. Ioannis Gkioulekas}}, Robotics, Carnegie Mellon} \hfill {\it August 2017 -- present} \\
\vspace*{-.13in}
\begin{list2}
\item Exploring optical imaging with coded coherence properties and its relationship to structured light
\item Working with a team of computational imaging experts across the US on a new method, called \href{http://seebelowtheskin.rice.edu/}{\textcolor{blue}{Computational Photo-Scatterography}}, to solve large-scale inverse problems in bioimaging, impacting medical and wellness applications ranging from wearables to non-invasive point-of-care devices.
\end{list2}

\vspace*{-0.1in}

{\bf A Bayesian framework for removing surgical smoke from laparoscopic images} \\
{\em Advisor: \href{https://www.cse.iitb.ac.in/~suyash}{\textcolor{blue}{Prof. Suyash Awate}}, CSE, IIT Bombay} \hfill {\it January 2015 -- June 2017} \\
\vspace*{-.13in}
\begin{list2}
\item Developed a Bayesian inference problem for jointly undoing the effect of surgical smoke, specularities and noise on laparoscopy images for better contrast and post-processing (like instrument tracking)
\item Tested this method extensively on simulated and real images yielding significant improvement over state of the art dehazing algorithms in terms of numerical and perceptual accuracy
\end{list2}

\vspace*{-0.1in}

{\bf Optimizing sensing for fast image acquisition} \hfill \textit{Master's Thesis} \\
{\em Advisor: \href{https://www.cse.iitb.ac.in/~ajitvr}{\textcolor{blue}{Prof. Ajit Rajwade}}, CSE, IIT Bombay} \hfill {\it December 2015 -- June 2017} \\
\vspace*{-.13in}
\begin{list2}
\item Worked on optimizing fast image acquisition models for compressive cameras
\item Found a case where conventional coherence-based optimization techniques make recovery worse, and proposed and successfully tested a new optimization criterion
\item Such acquisition techniques can significantly speed up medical imaging modalities like MR and CT

% \item Explored coherence minimization for optimizing reconstruction in compressed sensing in structured acquisition models like coded source separation and color image demosaicing 
% \item Discovered a sensing matrix structure where coherence minimization worsens reconstruction
% \item Concluded that looseness of worst-case coherence error bound causes this worsening 
% \item Demonstrated an average-case error-based design procedure, and showed reconstruction improvement in the structure where coherence fails
\end{list2}

% \vspace*{-0.1in}

% {\bf Super--resolution with Fourier ptychographic microscopy} \\
% {\em Advisor: \href{http://www.lvpei.org/our-team/our-team-ashutosh.php}{\textcolor{blue}{Dr. Ashutosh Richhariya}}, \href{http://www.lvpei.org/}{\textcolor{blue}{L. V. Prasad Eye Institute}}} \hfill {\it Winter 2014}
% \vspace*{-.13in}
% \textbf{} \\
% \begin{list2}
% \item Worked on understanding and implementing Fourier Ptychographic Microscopy for microscopy slides
% \item Analyzed possible extensions of this method to imaging reflective surfaces like the eye
% \end{list2}

% \vspace*{-0.13in}

% {\bf The IITB Mars Rover Project}
% \hfill {\it May 2013 -- June 2017} \\
% \vspace*{-.13in}
% \begin{list2}
% \item Built a prototype Mars rover capable of extra-terrestrial robotics with a rocker-bogie suspension
% \item Participated in a simulated Martian expedition in the Australian outback, at the \href{http://marssociety.org.au/article/arkaroola-mars-robot-challenge-spaceward-bound-expedition}{\textcolor{blue} {Arkaroola Mars Robot Challenge}} and at the Mars Society's \href{http://mdrs.marssociety.org/}{\textcolor{blue} {Mars Desert Research Station}}, Utah
% \end{list2}

%\vspace*{-0.13in}

% \section{\sc Research Internships} 
% %\vspace*{-0.13in}

% {\bf  \href{http://theairlab.org/}{\textcolor{blue}{The AIR Lab}}, \href{http://www.cmu.edu/}{\textcolor{blue}{Carnegie Mellon University}} \href{http://ri.cmu.edu/}{\textcolor{blue}{Robotics Institute}}} \\
% {\em Guide: \href{http://www.ri.cmu.edu/person.html?person_id=1397}{\textcolor{blue}{Prof. Sebastian Scherer}} \& \href{http://www.ri.cmu.edu/person.html?person_id=2128}{\textcolor{blue} {Stephen Nuske}}}\hfill\textit{Summer 2015} \\
% \vspace*{-.13in}
% \textbf{Stereo Odometry from a Downward-facing Stereo Camera on an Aerial Vehicle} \\
% \vspace*{-.01in}
% \begin{list2}
% \item Developed correlation-based tracking for aerial vehicles with a downward-facing stereo camera
% \item Estimated height, orientation using a robust homography fit between stereo pairs, position with rigid tracking, achieved better speed and height ranges than the Pixhawk camera without an inertial unit
% \end{list2}

% %\vspace*{-0.13in}

% {\bf \href{http://lcdm.astro.illinois.edu/}{\textcolor{blue} {Laboratory for Cosmological Data Mining}}, \href{http://www.illinois.edu/}{\textcolor{blue}{University of Illinois, Urbana--Champaign}}} \\
% {\em Guide: \href{http://www.astro.illinois.edu/people/bigdog}{\textcolor{blue}{Prof. Robert Brunner}}, under \href{https://www.google-melange.com/gsoc/homepage/google/gsoc2014}{\textcolor{blue} {Google Summer of Code}}} \hfill {\it Summer 2014} \\
% \vspace*{-.13in}
% \textbf{A Pixel-Level Machine Learning Method for Calculating Photometric Redshifts} \\
% \vspace*{-.01in}
% \begin{list2}
% \item Classified sources into galaxies, stars and background based on broad-band pixel flux
% \item Worked on creating an image extraction, alignment, cleaning, segmentation and learning pipeline on SDSS images and on performance improvement and got a reasonably good error rate
% \end{list2}

%\vspace*{-0.13in}

% \section{\sc Achievements \\and Awards}
% \begin{list2}
% \item Represented India at the \href{http://www.ioaa2012.ufrj.br/}{\textcolor{blue} {$6^\text{th}$ International Olympiad on Astronomy and Astrophysics}}, Brazil, 2012. Won a Gold Medal with International Rank 4 and a special prize for Best Data Analysis
% \item Represented India at the \href{http://www.ieso2011.unimore.it/}{\textcolor{blue} {$5^\text{th}$ International Earth Sciences Olympiad}}, Italy, 2011. Won a Bronze Medal and prizes for best performance in the Hydrosphere section and the team presentation
% \item Awarded the Undergraduate Research Award for an exceptional Master's project at IITB.
% \end{list2}

%\vspace*{-0.13in}

% \section{\sc Mentoring Experience}
% \textbf{Teaching Assistant: IITB}
% \begin{list1}
% \item[] CS663: Digital Image Processing \hspace{0.5cm} \href{https://www.cse.iitb.ac.in/~suyash}{\textcolor{blue}{Prof. S. Awate}} and \href{https://www.cse.iitb.ac.in/~ajitvr}{\textcolor{blue}{Prof. A. Rajwade}} \hfill{\textit{Autumn 2015-16}}
% \item[] CS736: Medical Image Processing \hspace{2cm} \href{https://www.cse.iitb.ac.in/~suyash}{\textcolor{blue}{Prof. S. Awate}}\hfill{\textit{Spring 2015-16}}
% \item[] EE638: Estimation and Identification \hspace{1.25cm} \href{https://www.ee.iitb.ac.in/course/~ee638/Navin}{\textcolor{blue}{Prof. N. Khaneja}}\hfill{\textit{Autumn 2016-17}}
% \item[] EE708: Information Theory and Coding \hspace{0.87cm} \href{https://www.ee.iitb.ac.in/wiki/faculty/bsraj}{\textcolor{blue}{Prof. S. B. Pillai}}\hfill{\textit{Spring 2016-17}}
% \end{list1}

% \vspace*{-0.1in}

% \textbf{Resource Person, Indian Astronomy Olympiad Programme} \hfill \textit{May '13, May '14, May '17} \\
% \vspace*{-.15in}
% \begin{list1}
% \item[] Involved in mentoring high-school students in Astronomy for their selection to the international Astronomy Olympiads, and in setting up challenging questions and evaluating students.
% \end{list1}

\vspace*{-0.13in}

\section{\sc Coursework} 
Medical Image Analysis, Digital Image Processing, Computer Vision, Computer Graphics, Machine Learning, Convex Optimization, Information Theory
% \begin{list1}
% \item[\strut\hspace{0.5cm}\hypertarget{crselst}{\textbf{IITB}}]\textit{Medical Image Analysis, Digital Image Processing, Computer Vision, Machine Learning, Convex Optimization, Information Theory}
% \item[\strut\hspace{0.5cm}\hypertarget{crselst}{\textbf{CMU}}] \textit{Computer Graphics*, Computer Vision*, Machine Learning} 
% \vspace{0.13in}
% \item[\strut\hspace{0.5cm}\textbf{IITB: Physics and Mathematics}]
% \vspace{0.05in}
% \item[]\textit{Astrophysics, General Relativity, Electromagnetic Waves, Classical Mechanics, Differential Equations, Linear Algebra, Complex Analysis, Calculus}
% \end{list1}

% \section{\sc Technical \\Skills} 
% \begin{tabular}{@{}p{1.3in}p{4.3in}}
% \textbf{Programming} & C/C++, Python, Bash, Matlab, SQL, \LaTeX \\  
% \vspace*{-0.06in}
% \textbf{Software Packages} & 
% \vspace*{-0.06in}
% ROS/Gazebo, OpenCV, The Point Cloud Library, Matplotlib \\ 
% \end{tabular}
%
% \section{\sc Other \\Interests}
% Other than my academic interests, I like biking, long walks, swimming, socializing, eating good food and trying to cook it. I especially enjoy classic rock music and people who enjoy my interests.
\vspace*{-0.13in} 

\section{\sc References}
\vspace{0.05in}
\begin{tabular}{@{}p{3in}p{3in}}
\textbf{Prof. Ioannis Gkioulekas}, Robotics & \textbf{Prof. Suyash Awate}, CSE \\ 
Carnegie Mellon $|$ \href{mailto:igkioule@cs.cmu.edu}{\textcolor{blue}{E--Mail}} $|$ \href{http://www.cs.cmu.edu/~igkioule/}{\textcolor{blue}{Webpage}} & IIT Bombay $|$ \href{mailto:suyash@cse.iitb.ac.in}{\textcolor{blue}{E--Mail}} $|$ \href{https://www.cse.iitb.ac.in/~suyash}{\textcolor{blue}{Webpage}} \\
\end{tabular}
\vspace{-0.15in}

\begin{tabular}{@{}p{3in}p{3in}}
\textbf{Prof. Ajit Rajwade}, CSE & \textbf{Dr. Aniket Sule}, Astronomy \\
IIT Bombay $|$ \href{mailto:ajitvr@cse.iitb.ac.in}{\textcolor{blue}{E--Mail}} $|$ \href{https://www.cse.iitb.ac.in/~ajitvr}{\textcolor{blue}{Webpage}} & TIFR $|$ \href{mailto:anikets@hbcse.tifr.res.in}{\textcolor{blue}{E--Mail}} $|$ \href{http://www.hbcse.tifr.res.in/people/academic/aniket-sule}{\textcolor{blue}{Webpage}} \\
\end{tabular}
\vspace{-0.15in}

%\begin{tabular}{@{}p{3in}p{3in}}
%\textbf{Dr. Sebastian Scherer}, Robotics Institute & \textbf{Dr. Ashutosh Richhariya}, Ophthalmic Biophysics \\ 
%CMU $|$ \href{mailto:basti@andrew.cmu.edu}{\textcolor{blue}{E--Mail}} $|$ \href{http://www.ri.cmu.edu/person.html?person_id=1397}{\textcolor{blue}{Webpage}} & LVPEI $|$ \href{mailto:ashutosh@lvpei.org}{\textcolor{blue}{E--Mail}} $|$ \href{http://www.lvpei.org/our-team/our-team-ashutosh.php}{\textcolor{blue}{Webpage}} \\
%\end{tabular}
%\vspace{-0.15in}

% \begin{tabular}{@{}p{3in}p{3in}}
% \textbf{Prof. Rajbabu Velmurugan}, EE & \textbf{Dr. Manojendu Choudhury}, Astrophysics \\
% IITB $|$ \href{mailto:rajbabu@ee.iitb.ac.in}{\textcolor{blue}{E--Mail}} $|$ \href{https://www.ee.iitb.ac.in/web/faculty/homepage/rajbabu}{\textcolor{blue}{Webpage}} & UM--DAE CBS $|$ \href{mailto:manojendu@cbs.ac.in}{\textcolor{blue}{E--Mail}} $|$ \href{http://www.cbs.ac.in/people/physics-faculty/manojendu-choudhury}{\textcolor{blue}{Webpage}} \\
% \end{tabular}
% \vspace{-0.15in}

\end{resume}
\end{document}
