\documentclass[11pt,a4paper,roman]{moderncv} % Font sizes: 10, 11, or 12; paper sizes: a4paper, letterpaper, a5paper, legalpaper, executivepaper or landscape; font families: sans or roman

\moderncvstyle{casual} % CV theme - options include: 'casual' (default), 'classic', 'oldstyle' and 'banking'
\moderncvcolor{blue} % CV color - options include: 'blue' (default), 'orange', 'green', 'red', 'purple', 'grey' and 'black'

\usepackage{color}
\usepackage{lipsum} % Used for inserting dummy 'Lorem ipsum' text into the template

\usepackage[scale=0.80]{geometry} % Reduce document margins
\setlength{\hintscolumnwidth}{1.9cm} % Uncomment to change the width of the dates column
%\setlength{\makecvtitlenamewidth}{10cm} % For the 'classic' style, uncomment to adjust the width of the space allocated to your name

%----------------------------------------------------------------------------------------
%	NAME AND CONTACT INFORMATION SECTION
%----------------------------------------------------------------------------------------

\firstname{Alankar} % Your first name
\familyname{Kotwal} % Your last name

% All information in this block is optional, comment out any lines you don't need
\title{Detailed R\'esum\'e}
\email{alankar.kotwal@iitb.ac.in}
\homepage{alankarkotwal.github.io}{alankarkotwal.github.io}
\mobile{+91-9969678123}
\extrainfo{\url{www.github.com/alankarkotwal}}
\photo[70pt][0.4pt]{pic.jpg}

\begin{document}
\makecvtitle

\section{Education}

\cventry{2012--2017}{Dual Degree, B. Tech and M.Tech in Electrical Engineering}{\newline Indian Institute of Technology}{Bombay}{\textit{CPI -- 8.78/10}}{Communication and Signal Processing, Minor: Computer Sciences and Engineering}
\cventry{2010--2012}{Intermediate Examination}{\newline Ratanbai Walbai Junior College of Science}{Mumbai}{\textit{Percentage -- 93.83}}{}
\cventry{2001--2010}{Matriculation}{\newline SVPT's Saraswati Vidyalaya}{Thane}{\textit{Percentage -- 95.27}}{}

\section{Achievements}
\cventry{Aug 2012}{\textbf{Gold Medal, International Olympiad on Astronomy \& Astrophysics}}{\newline Brazil} {International Rank 4, Special Prize for Best Data Analysis}{}{}
\cventry{Sep 2011}{\textbf{Bronze Medal, International Earth Sciences Olympiad}}{\newline Italy} {Special Prize for Best Performance in Hydrosphere section}{}{}
\cventry{Apr 2012}{All India Rank 105}{IIT-JEE}{\newline among around 5,90,000 participants for entrance to the IITs}{}{}
\cventry{2009--2012}{Olympiad Orientation-cum-Selection Camps}{}{\newline Selected for the following camps, among the top 30 students in India (Astronomy: 2012 \& 2010, Earth Sciences: 2011, Junior Sciences: 2010 \& 2009)}{}{}
\cventry{2010}{Kishore Vaigyanik Protsahan Yojana Scholarship}{}{\newline Awarded by the Government of India to students interested in research}{}{}
\cventry{2008}{National Talent Search Examination Scholarship}{}{\newline Awarded by the Government of India to students interested in research}{}{}
\cventry{2011--2012}{Infosys Award for Olympiad Medallists}{}{}{}{}
\cventry{Dec 2013}{Inter-IIT Messier Marathon}{}{\newline Secured IIT Bombay the second position by putting on board 72 messier objects including the entire Virgo cluster of galaxies}{}{}
\cventry{2013}{Other competitions}{}{\newline Won the Innovation Cell recruitment contest for freshmen and the Astronomy Quiz conducted by the Astronomy Club, IITB in 2012 and BITS Goa in 2013}{}{}

\newpage

\section{Experience: Electrical Engineering and Computer Sciences}

\cventry{Summer 2015}{Research Internship, The Robotics Institute}{\newline Stereo odometry from a downward-facing stereo camera on a vehicle}{\newline Prof. Sebastian Scherer and Stephen Nuske}{Carnegie Mellon University}{
    \begin{itemize}
        \item{Explored correlation-based stereo odometry for quadcopter localisation applications}
        \item{Implemented gradient-descent for tracking and homography fits to obtain 6-DoF pose}
    \end{itemize}
}

\cventry{Summer 2014}{Google Summer of Code}{\newline A New Pixel-Level Method for Determination of Photometric Redshifts}{\newline Prof. R. Brunner \& M. Kind, Laboratory for Cosmological Data Mining}{UIUC}{
\begin{itemize}
    \item{Used SDSS photometry to extract pixel information for machine learning algorithms}
    \item{Worked on parallel programming and performance enhancement}
    \item{Validated the approach and got consistent predictions for redshifts in the testing set}
\end{itemize}
}

\cventry{Spring 2015}{Laparoscopic Image Dehazing}{\newline Using the Dark Channel Prior to De-Haze Laparoscopy Images}{\newline Prof. S. Awate, Department of Computer Sciences}{IIT Bombay}{
\begin{itemize}
    \item{Used the statistical properties of natural images to remove haze effects}
    \item{Working on an optimization model for the same process}
\end{itemize}
\textit{Paper "Joint De-smoking and Denoising of Laparoscopy Images Using Probabilistic Graphical Modeling With Bayesian Inference" submitted to the International Symposium on Biomedical Imaging (ISBI), 2016}
}

\cventry{2013--2015}{Computer Vision, The IITB Mars Rover Team}{\newline A Student Initiative at IITB}{}{}{
\begin{itemize}
    \item{Exploring stereo vision for autonomous navigation and obstacle avoidance}
    \item{Implementation of the rover software stack on ROS}
    \item{Hardware interfacing for peripherals on-board and debugging}
\end{itemize}
}

\cventry{Summer 2014}{The Arkaroola Mars Robot Challenge}{\newline A joint venture of the Mars Society Australia and Saber Astronautics}{}{}{
\begin{itemize}	
\item{Tested the Mars Rover prototype in the harsh conditions of the Australian outback}
\item{Participated in a series of exercises in Mars operations research conducted by Saber Astronautics including simulated extra-vehicular activities in simulated space-suits}
\end{itemize}
\textit{Featured in Clarke et al., "Field Robotics, Astrobiology and Mars Analogue Research on the Arkaroola Mars Robot Challenge Expedition", Australian Space Research Conference}
}

\cventry{Spring 2015}{Stereo Visual Odometry from Pointclouds}{\newline Using point-set registration for localization}{\newline Prof. A. Rajwade, Department of Computer Sciences}{IIT Bombay}{
\begin{itemize}
    \item{Explored kernel-correlation maximisation for point-set registration}
    \item{Implemented coherent point drift for pointclouds in C++}
\end{itemize}
}

\cventry{Autumn 2014}{Gravitational Lens Identification Using Image Processing Techniques}{\newline A PCA-based Method for Identifying Lenses in Databases}{\newline Prof. A. Rajwade and S. Awate, Department of Computer Sciences}{IIT Bombay}{
\begin{itemize}
\item{Improvised on source-subtraction algorithms for lens subtraction}
\item{Implemented the algorithm in Matlab and got a good identification rate for lenses}
\end{itemize}
}

\cventry{Autumn 2014}{Microprocessor Design}{\newline Design, Implementation and Validation of Three Processors in Verilog}{\newline Prof. V. Singh, Department of Electrical Engineering}{IIT Bombay}{
\begin{itemize}
\item{Designed and simulated a pipelined processor with the Little Computer Architecture}
\item{Designed, implemented and tested a multi-cycle RISC processor using the LC-3b ISA}
\end{itemize}
}

\newpage

\section{Experience: Astronomy and Astrophysics}

\cventry{Dec 2013}{National Initiative for Undergraduate Studies -- Astronomy}{\newline An X-Ray Study of Black Hole Candidate X Norma X-1}{\newline Prof. Manojendu Choudhury}{Center for Basic Sciences, University of Mumbai}{
\begin{itemize}
\item{Analysed timing information from RXTE to detect quasi-periodic oscillations}
\item{Fitted obtained spectra \& observed unusual oscillations in the inner radius}
\end{itemize}
}

\cventry{Dec 2012}{National Initiative for Undergraduate Studies -- Astronomy}{\newline Estimation of Photometric Redshifts Using Machine Learning Techniques}{\newline Prof. Ninan Sajeeth Philip}{IUCAA, Pune}{
\begin{itemize}
\item{Estimated redshifts from colour information obtained from SDSS using neural networks}
\item{Worked on generation of training data from available data by redshifting spectra}
\end{itemize}
}

\section{Positions of Responsibility}
\cventry{Autumn 2015}{Teaching Assistant}{CS663 -- Digital Image Processing}{\newline Prof. A. Rajwade and S. Awate, Department of Computer Sciences}{IIT Bombay}{
\begin{itemize}
    \item{Involved in setting \& evaluating assignments \& exams}
    \item{Mentoring students with the course material \& projects}
\end{itemize}
}

\cventry{May 2013 May 2014}{Resource Person}{\newline Indian National Astronomy Olympiad Programme}{}{HBCSE -- TIFR, Mumbai}{
\begin{itemize}
    \item{Student facilitator for the Astronomy Camp for mentoring \& evaluating students}
    \item{Involved in generating problems for the Indian National Astronomy Olympiad}
\end{itemize}
}

\section{Relevant Skills}
\cvitem{\textbf{Languages}}{C/C++, Python, Shell Scripting, Matlab, SQL, HTML, PHP, \LaTeX}
\cvitem{\textbf{Special Software}}{ROS/Gazebo, OpenCV, The Point Cloud Library, SPICE Circuit Simulation, EAGLE PCB Design, SolidWorks CAD, AutoCAD, LabView}
\cvitem{\textbf{Science Software}}{Python packages: NumPy, SciPy and Matplotlib, GNUPlot, Scikit-learn, Astropy, SExtractor, SDSS tools}
\cvitem{\textbf{Hardware}}{\textit{Microprocessor Architectures}: 8051, 8085, AVR and PIC, CPLDs and FPGAs, \textit{Embedded Platforms}: Arduino, RaspberryPi, standard digital logic families}

\section{Relevant Courses Undertaken}

\cvitem{\textbf{Computer Sciences}}{Computer Graphics, Computer Vision, Algorithms for Medical Image Processing, Machine Learning, Convex Optimisation, Digital Image Processing, Design and Analysis of Algorithms, Data Structures and Algorithms, Discrete Mathematics}

\cvitem{\textbf{Electrical Engg}}{Estimation and Identification, Speech Processing, Digital Signal Processing, Controls, Probability and Random Processes, Digital Communication, Communication Systems, Microprocessors, Signals and Systems, Digital and Analog Systems, Electronic Devices and Circuits, Network Theory}

\cvitem{\textbf{Physics \& Maths}}{The General Theory of Relativity, Electromagnetic Waves, Electricity and Magnetism, Classical Mechanics, Differential Equations, Linear Algebra, Complex Analysis, Calculus}

%\section{Research Interests}
%\cventry{}{Electrical Engineering and Computer Sciences}{}{}{}{
%\begin{itemize}
%    \item{Using computational Fourier Optics for imaging resolution improvement}
%    \item{3D shape reconstruction using computer vision techniques}
%    \item{Robot navigation using stereo vision and structure from motion}
%    \item{Efficient algorithms for robot navigation using geometry of visual field}
%    \item{Processor architecture}
%    \item{Hardware description and simulation}
%\end{itemize}
%}

%\cventry{}{Astronomy and Astrophysics}{}{}{}{
%\begin{itemize}
%    \item{Cosmology and the large-scale structure of the universe}
%    \item{Stellar populations, structure and evolution}
%    \item{Applications of computer vision to astronomy}
%    \item{Data mining and its applications for handling astronomical data}
%\end{itemize}
%}

%\cventry{}{Things I'd like to do}{}{}{}{
%\begin{itemize}
%    \item{Logic minimization}
%    \item{Operations research in relation to Mars missions}
%\end{itemize}
%}

\end{document}
