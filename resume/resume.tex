\documentclass[margin,line]{res}

\usepackage{hyperref}
\usepackage{amsmath}
\usepackage{textcomp}
\usepackage{color}
\oddsidemargin -.5in
\evensidemargin -.5in
\topmargin -0.3in
\textheight 9.5in
\textwidth=6.0in
\itemsep=0in
\parsep=0in
% if using pdflatex:
%\setlength{\pdfpagewidth}{\paperwidth}
%\setlength{\pdfpageheight}{\paperheight} 

\newenvironment{list1}{
  \begin{list}{\ding{113}}{%
      \setlength{\itemsep}{0in}
      \setlength{\parsep}{0in} \setlength{\parskip}{0in}
      \setlength{\topsep}{0in} \setlength{\partopsep}{0in} 
      \setlength{\leftmargin}{0.17in}}}{\end{list}}
\newenvironment{list2}{
  \begin{list}{$\bullet$}{%
      \setlength{\itemsep}{0in}
      \setlength{\parsep}{0in} \setlength{\parskip}{0in}
      \setlength{\topsep}{0in} \setlength{\partopsep}{0in} 
      \setlength{\leftmargin}{0.2in}}}{\end{list}}

\tolerance=1
\emergencystretch=\maxdimen
\hyphenpenalty=10000
\hbadness=10000

\pagenumbering{gobble}

\begin{document}

%\name{\textbf{\huge{Alankar Kotwal}} \vspace*{.05in} \newline  {\sc Senior Undergraduate}\vspace*{.1in}}
\name{\textbf{\huge{Alankar Kotwal}} \vspace*{.1in}}

\begin{resume}
%\section{\sc Contact Information}
%\vspace{.05in}
%\begin{tabular}{@{}p{2.9in}p{.5in}p{3in}}
%Electrical Engineering, IIT Bombay & \multicolumn{1}{r}{\it Phone:}  &(+91) 996 967 8123 \\            
%281, Hostel 09, IIT Bombay &\multicolumn{1}{r}{\it E--Mail:}& \href{mailto:alankar.kotwal@iitb.ac.in}{\textcolor{blue}{alankar.kotwal@iitb.ac.in}} \\
%Powai, Mumbai, India 400 076 & \multicolumn{1}{r}{\it Webpage:} &\href{http://alankarkotwal.github.io/}{\textcolor{blue}{alankarkotwal.github.io}} \\     
%\end{tabular}

%\vspace*{-0.13in}

\section{\sc Research Interests}
I am passionate about Computer and Medical Vision, Machine Learning, Optimization, Estimation Theory, Astrophysics and Cosmology. I also like Robotics, Networks \& Security and Graphics.

\vspace*{-0.13in}

\section{\sc Education}
{\bf \href{http://www.iitb.ac.in/}{\textcolor{blue}{Indian Institute of Technology Bombay}}}, Mumbai, India \hfill {\it July 2012 -- Present} \\
\vspace*{-.1in}
\begin{list1}
\item[] Fifth Year, Dual Degree (Bachelor \& Master of Technology), Department of \href{http://www.ee.iitb.ac.in/}{\textcolor{blue}{Electrical Engineering}}
\item[] Specialization: {\em Communication and Signal Processing}, {\bf CGPA: 8.92/10.00}
\end{list1}

\vspace*{-0.13in}

\section{\sc Publications}
\begin{list2}
\item Kotwal, A., Bhalodia, R., Awate, S., {\em Joint Desmoking and Denoising of Laparoscopy Images} (oral), Proc. of the \href{http://biomedicalimaging.org/2016/}{\textcolor{blue} {International Symposium on Biomedical Imaging}}, 2016. Paper \href{http://alankarkotwal.github.io/pubs/lap-dehazing.pdf}{\textcolor{blue} {here}}.
\item Clarke, J. D. A., Held, J. M., Dahl, A. {\em et al.}, {\em Field Robotics, Astrobiology and Mars Analogue Research on the Arkaroola Mars Robot Challenge Expedition}, Proc. of the \href{http://www.nssa.com.au/14asrc/14ASRC-proceedings.zip}{\textcolor{blue} {14th Australian Space Research Conference}}, 2014. Paper \href{http://alankarkotwal.github.io/pubs/Arkaroola.pdf}{\textcolor{blue} {here}}.
\end{list2}

\vspace*{-0.13in}

\section{\sc Research Internships} 

{\bf  \href{http://theairlab.org/}{\textcolor{blue}{The AIR Lab}}, \href{http://www.cmu.edu/}{\textcolor{blue}{Carnegie Mellon University}} \href{http://ri.cmu.edu/}{\textcolor{blue}{Robotics Institute}}} \\
{\em Guide: \href{http://www.ri.cmu.edu/person.html?person_id=1397}{\textcolor{blue}{Prof. Sebastian Scherer}} \& \href{http://www.ri.cmu.edu/person.html?person_id=2128}{\textcolor{blue} {Stephen Nuske}}}\hfill\textit{Summer 2015} \\
\vspace*{-.13in}
\textbf{Stereo Odometry From A Downward-Facing Stereo Camera On A Vehicle} \\
\vspace*{-.01in}
\begin{list2}
\item Developed correlation-based tracking for aerial vehicles with a downward-facing stereo camera
\item Estimated height, pitch and roll jointly using a robust gradient-descent homography fit between stereo pairs, and position with rigid tracking across frames
\item Achieved performance comparable to, maximum speeds and height ranges better than the standard Pixhawk PX4FLOW camera without an inertial unit
\end{list2}

\vspace*{-0.13in}

{\bf \href{http://lcdm.astro.illinois.edu/}{\textcolor{blue} {Laboratory for Cosmological Data Mining}}, \href{http://www.illinois.edu/}{\textcolor{blue}{University of Illinois, Urbana--Champaign}}} \\
{\em Guide: \href{http://www.astro.illinois.edu/people/bigdog}{\textcolor{blue}{Prof. Robert Brunner}}, under \href{https://www.google-melange.com/gsoc/homepage/google/gsoc2014}{\textcolor{blue} {Google Summer of Code}}} \hfill {\it Summer 2014} \\
\vspace*{-.13in}
\textbf{A Pixel-Level Machine Learning Method for Calculating Source Redshifts} \\
\vspace*{-.01in}
\begin{list2}
\item Used broad-band pixel energies from faint sources extracted from SDSS (as a proxy to their entire spectrum) as features for a machine learning algorithm to calculate redshifts
\item Accomplished classification of sources into galaxies, stars and background based on pixel features
\item Worked on creating an image extraction, alignment, cleaning, segmentation and learning pipeline on SDSS images and on performance improvement and got a reasonably good error rate
\end{list2}

\vspace*{-0.13in}

\section{\sc Research Projects}

{\bf A New Bayesian Framework For Laparoscopic Image Dehazing and Denoising} \\
{\em Guide: \href{https://www.cse.iitb.ac.in/~suyash}{\textcolor{blue}{Prof. Suyash Awate}}, CSE, IITB} \hfill {\it January 2015 -- Present} \\
\vspace*{-.13in}
\begin{list2}
\item Developed a Bayesian inference problem for jointly undoing the effect of surgical smoke and noise on laparoscopy images for better contrast and post-processing (like segmentation and tracking)
\item Tested this method extensively on simulated and real images yielding significant improvement over state of the art dehazing algorithms in terms of numerical and perceptual accuracy
\item Surveyed laparoscopy experts about quality of our results compared to the existing algorithms and found a statistically significant trend that this method yields superior results
\end{list2}

\vspace*{-0.13in}

{\bf Coded Source Separation for Compressed Video Recovery} \hfill \textit{Dual Degree Thesis} \\
{\em Guide: \href{https://www.cse.iitb.ac.in/~ajitvr}{\textcolor{blue}{Prof. Ajit Rajwade}}, CSE, and \href{https://www.ee.iitb.ac.in/wiki/faculty/rajbabu}{\textcolor{blue}{Prof. V. Rajbabu}}, EE, IITB} \hfill {\it December 2015 -- Present} \\
\vspace*{-.13in}
\begin{list2}
\item Studied applications of the principles of compressed sensing to video for compression along time
\item Currently trying to relax the requirement of a dictionary on both space and time and the requirement of strictly smooth motion using a source--separation approach to this problem
\item Aim to design positive [0, 1]--uniform sensing matrices with low mutual coherence, making them ideal for compressed video using the source--separation approach
\end{list2}

\vspace*{-0.13in}

{\bf The IITB Mars Rover Project}
%\\
%{\em Guide: \href{http://www.aero.iitb.ac.in/~pjguru/}{\textcolor{blue}{Prof. PJ Guruprasad}}, Aerospace, IITB}
\hfill {\it May 2013 -- Present} \\
\vspace*{-.13in}
\begin{list2}
\item Aim to build a prototype Mars rover capable of extra-terrestrial robotics, currently have a rover with a rocker-bogie suspension and novel air-filled beach tires
\item Designed power, logic and communication circuits for on-board control and interfaced peripherals for perception and actuation, currently developing localization and autonomous navigation and exploring he role of machine vision for automating rover operations
\item Participated in a simulated Martian expedition in the Australian outback, at the \href{http://marssociety.org.au/article/arkaroola-mars-robot-challenge-spaceward-bound-expedition}{\textcolor{blue} {Arkaroola Mars Robot Challenge}} and at the Mars Society's \href{http://mdrs.marssociety.org/}{\textcolor{blue} {Mars Desert Research Station}}, Utah
\end{list2}

\vspace*{-0.13in}

\section{\sc Course \\Projects}
{\bf Improved Methods for Compressed Sensing Recovery} \hfill {\it CS709: Convex Optimization} \\
{\em Guide: \href{https://www.cse.iitb.ac.in/~ganesh/}{\textcolor{blue}{Prof. Ganesh Ramakrishnan}}, CSE, IITB \hfill Autumn 2015-16} \\
\vspace*{-.15in}
\begin{list1}
\item[] Using convex approximations to the compressed sensing recovery problem, we reconstructed near-exact versions of images at extremely low compressions, with proofs of correctness. Code \href{https://github.com/alankarkotwal/cs-rank-minimization}{\textcolor{blue} {here}}.
\end{list1}

\vspace*{-0.13in}

{\bf Hidden Markov Model Part-of-Speech Tagging} \hfill \textit{EE638: Estimation and Identification} \\
{\em Guide: \href{https://www.ee.iitb.ac.in/course/~ee638/Navin}{\textcolor{blue}{Prof. Navin Khaneja}}, EE, IITB \hfill Autumn 2015-16} \\
\vspace*{-.15in}
\begin{list1}
\item[] We implemented part-of-speech tagging with support for unknown words. An error rate of around 5\% and capabilities of the system to discern context were observed.
\end{list1}

\vspace*{-0.13in}

{\bf Laparoscopic Image Dehazing With Dark Channel Prior} \hfill \textit{CS736: Medical Image Processing} \\
{\em Guide: \href{https://www.cse.iitb.ac.in/~suyash}{\textcolor{blue}{Prof. Suyash Awate}}, CSE, IITB \hfill Spring 2014-15} \\
\vspace*{-.15in}
\begin{list1}
\item[] We applied the Dark Channel Prior method for landscape image dehazing to surgical smoke--affected laparoscopic images, accelerated it in time and got good results.
\end{list1}

\vspace*{-0.13in}

{\bf Stereo Odometry Via Point Cloud Registration} \hfill \textit{CS763: Computer Vision} \\
{\em Guide: \href{https://www.cse.iitb.ac.in/~ajitvr}{\textcolor{blue}{Prof. Ajit Rajwade}}, CSE, IITB \hfill Spring 2014-15} \\
\vspace*{-.15in}
\begin{list1}
\item[] Maximizing kernel density correlation with gradient-ascent and coherent point drift, we registered pointclouds and observed good convergence behavior for small transformations.
\end{list1}

\vspace*{-0.13in}

{\bf Gravitational Lens Separation With PCA} \hfill \textit{CS663: Digital Image Processing} \\
{\em Guide: \href{https://www.cse.iitb.ac.in/~suyash}{\textcolor{blue}{Prof. Suyash Awate}} and \href{https://www.cse.iitb.ac.in/~ajitvr}{\textcolor{blue}{Prof. Ajit Rajwade}}, CSE, IITB \hfill Autumn 2014-15} \\
\vspace*{-.15in}
\begin{list1}
\item[] Gravitationally lensed images of galaxies have rare arc-like artifacts that can be used to calculate the mass of the lens. We used Anscombe denoising followed by PCA to build a basis for galaxy images and used the top few eigengalaxies to subtract sources and detect arcs.
\end{list1}

\vspace*{-0.13in}

%{\bf Processor Design} \hfill \textit{EE309: Microprocessors} \\
%{\em Guide: \href{https://www.ee.iitb.ac.in/~viren/}{\textcolor{blue}{Prof. Virendra Singh}}, EE, IITB \hfill Autumn 2014-15} \\
%\vspace*{-.15in}
%\begin{list1}
%\item[] We designed, simulated and implemented (on a DE0-Nano board from Terasic) a multi-cycle RISC processor following the LC-3b ISA. Following this, we designed and simulated a pipelined RISC processor using the Little Computer Architecture.
%\end{list1}
%
%\vspace*{-0.13in}

\section{\sc Astrophysics \\Projects}
{\bf Detection of Short Gamma-ray Bursts from Astrosat Data} \hfill \textit{PH426: Astrophysics} \\
{\em Guide: \href{https://sites.google.com/site/vikramrentalahome/}{\textcolor{blue}{Prof. Vikram Rentala}}, \textit{PH, IITB} \hfill Spring 2015-16} \\
\vspace*{-.15in}
\begin{list1}
\item[] Among the open problems and new datasets in the field, we tackle detecting short gamma-ray bursts from data acquired by the CZTI Hard X-Ray Imager on board the Astrosat.
\end{list1}

\vspace*{-0.13in}

{\bf Processing and Inference from CCD Images} \hfill \textit{\href{http://nius.hbcse.tifr.res.in/}{\textcolor{blue} {NIUS, Astronomy}}} \\
{\em Guide: \href{http://http://manuu.ac.in/deptphysc_faclty.php/}{\textcolor{blue}{Prof. Priya Hasan}}, \href{http://manuu.ac.in/}{\textcolor{blue} {MANUU, Hyderabad}} \hfill December 2015} \\
\vspace*{-.15in}
\begin{list1}
\item[] We analyzed raw data for the globular cluster NGC2419 taken at the \href{http://www.iiap.res.in/iao/cycle.html}{\textcolor{blue} {HCT}}, post-processed it and extracted the variation of magnitudes of stars in the cluster on the scale of a day. Code \href{hhttps://github.com/alankarkotwal/ngc2419-variables}{\textcolor{blue} {here}}.
\end{list1}

\vspace*{-0.13in}

{\bf An X-Ray Study of Black Hole Candidate X Norma X-1} \hfill \textit{\href{http://nius.hbcse.tifr.res.in/}{\textcolor{blue} {NIUS, Astronomy}}} \\
{\em Guide: \href{http://cbs.ac.in/people/visiting-scientists/manojendu-choudhury}{\textcolor{blue}{Prof. Manojendu Choudhury}}, \href{http://cbs.ac.in/}{\textcolor{blue} {Center for Basic Sciences}} \hfill December 2013} \\
\vspace*{-.15in}
\begin{list1}
\item[] We analyzed spectral data for the X-Ray Binary 4U 1630-47, in a period that corresponds to an outburst in the source for various system parameters like internal radius, temperature and so on.
\end{list1}

\vspace*{-0.13in}

{\bf Estimation of Photometric Redshifts Using Machine Learning} \hfill \textit{\href{http://nius.hbcse.tifr.res.in/}{\textcolor{blue} {NIUS, Astronomy}}} \\
{\em Guide: \href{http://www.iucaa.ernet.in/~nspp/}{\textcolor{blue}{Prof. Ninan Sajeeth Philip}}, \href{http://www.iucaa.ernet.in/}{\textcolor{blue} {IUCAA}}, Pune \hfill December 2012} \\
\vspace*{-.15in}
\begin{list1}
\item[] Here, we trained a neural network to calculate photometric redshifts and used SDSS data and its redshifted versions to train it, getting good predictions for redshift.
\end{list1}

\vspace*{-0.13in}

\section{\sc Achievements \\and Awards}
\begin{list2}
\item[\strut\hspace{0.5cm}\textbf{Olympiads and Competitive Exams}]
\item Represented India at the \href{http://www.ioaa2012.ufrj.br/}{\textcolor{blue} {$6^{th}$ International Olympiad on Astronomy and Astrophysics}}, Brazil, 2012. Won a Gold Medal with International Rank 4 and a special prize for Best Data Analysis
\item Represented India at the \href{http://www.ieso2011.unimore.it/}{\textcolor{blue} {$5^{th}$ International Earth Sciences Olympiad}}, Italy, 2011. Won a Bronze Medal and prizes for best performance in the Hydrosphere section and the team presentation
\item  Secured All India Rank (AIR) $105$ in \href{https://en.wikipedia.org/wiki/Indian_Institute_of_Technology_Joint_Entrance_Examination}{\textcolor{blue}{IIT-JEE}} amongst $1.1$ million candidates
\vspace{0.05in}
\item[\strut\hspace{0.5cm}\textbf{Scholarships}]
\item Awarded \href{http://www.kvpy.iisc.ernet.in/main/index.htm}{\textcolor{blue}{KVPY Scholarship}} $2011$ by Dept. of Science and Technology, Govt. of India
\item Awarded \href{http://www.ncert.nic.in/programmes/talent_exam/index_talent.html}{\textcolor{blue}{NTSE Scholarship}} $2008$ by NCERT, Govt. of India
%\item[\strut\hspace{0.5cm}\textbf{Competitions}]
%\item Secured IIT Bombay the second position by putting on board 72 Messier objects including the entire Virgo cluster of galaxies in the \href[page=2]{https://www.iitbombay.org/iitb_dean_acr/february-newsletter-2014/2nd\%20Annual\%20Inter.pdf}{\textcolor{blue} {Inter-IIT Messier Marathon, 2014}}
%\item Won the Astronomy Quiz conducted by the Astronomy Club, IIT Bombay, 2012
\end{list2}

\vspace*{-0.13in}

%\section{\sc Talks and \\Seminars}
%
%{\bf Template-Based Stereo Odometry} \hfill {\em Invited Talk} \\
%{\em \href{http://theairlab.org/}{\textcolor{blue}{The AIR Lab}}, \href{http://www.cmu.edu/}{\textcolor{blue}{Carnegie Mellon University}} \hfill July 2015} \\
%\vspace*{-.15in}
%\begin{list1}
%\item[] Here, I presented results from my 2015 summer internship to my group at Carnegie Mellon University. The talk included a detailed description of the method used, comparisons of the results with ground-truth and stress-tests on the method. Presentation \href{http://alankarkotwal.github.io/intern_presentation.pptx}{\textcolor{blue} {here}}.
%\end{list1}
%
%{\bf \href{http://www.stab-iitb.org/krittika/the-cosmic-ladder-distance}{\textcolor{blue} {The Cosmic Distance Ladder}}} \hfill {\em Invited Talk} \\
%{\em \href{http://www.stab-iitb.org/krittika/}{\textcolor{blue} {Krittika -- The Astronomy Club, IIT Bombay}} \hfill September 2014, February 2016} \\
%\vspace*{-.15in}
%\begin{list1}
%\item[] This open-to-all talk is a journey climbing the Cosmic Distance Ladder, which is a sequence of steps, each building on the previous step's results, for calculating distances in the universe. We begin with solar system distances, and end at huge distances where the only real option is to use photometric redshifts. Presentation \href{http://alankarkotwal.github.io/CosmicDistanceLadder.pptx}{\textcolor{blue} {here}}.
%\end{list1}


%\section{\sc Mentoring Experience}
%\textbf{Teaching Assistant}
%\begin{list1}
%\item[]CS663: Digital Image Processing \hspace{0.5cm} \href{https://www.cse.iitb.ac.in/~suyash}{\textcolor{blue}{Prof. S. Awate}} and \href{https://www.cse.iitb.ac.in/~ajitvr}{\textcolor{blue}{Prof. A. Rajwade}} \hfill{\textit{Autumn 2015-16}}
%\item[]CS736: Medical Image Processing \hspace{2cm} \href{https://www.cse.iitb.ac.in/~suyash}{\textcolor{blue}{Prof. S. Awate}}\hfill{\textit{Spring 2015-16}}
%\end{list1}
%
%\vspace*{-0.1in}
%
%\textbf{Resource Person, Indian Astronomy Olympiad Programme} \hfill \textit{May 2013, May 2014} \\
%\vspace*{-.15in}
%\begin{list1}
%\item[] Selected twice as a resource person for the Indian Astronomy Olympiad Camp, for their selection to the international Astronomy Olympiads. Involved in mentoring students ranging from the $9^{th}$ to the $12^{th}$ grades in Astronomy, and in setting up challenging questions and evaluating students.
%\end{list1}

%\section{\sc Key \\Coursework} 
%\begin{list1}
%\item[\strut\hspace{0.5cm}\hypertarget{crselst}{\textbf{Computer Sciences and Engineering}}]
%\item[]\textit{Machine Learning, Convex Optimization, Computer Vision, Medical Image Processing, Digital Image Processing, Computer Graphics, Computer Networks, Algorithms, Discrete Mathematics}
%\item[\strut\hspace{0.5cm}\textbf{Electrical Engineering}]
%\vspace{0.05in}
%\item[]\textit{Estimation and Identification, Adaptive Signal Processing, Speech Processing, Matrix Computations, Information Theory, Advanced Probability, Communication Networks}
%\item[\strut\hspace{0.5cm}\textbf{Physics and Mathematics}]
%\vspace{0.05in}
%\item[]\textit{Astrophysics, The General Theory of Relativity, Electromagnetic Waves, Electricity \& Magnetism, Classical Mechanics, Differential Equations, Linear Algebra, Complex Analysis, Calculus}
%\end{list1}

\section{\sc Technical \\Skills} 
\begin{tabular}{@{}p{1.3in}p{4.3in}}
\textbf{Programming} & C/C++, Python, Bash, Matlab, Verilog, SQL, HTML, PHP, \LaTeX \\  
\vspace*{-0.06in}
\textbf{Software Packages} & 
\vspace*{-0.06in}
ROS/Gazebo, OpenCV, The Point Cloud Library, SPICE Circuit Simulation, EAGLE PCB Design, SolidWorks, AutoCAD, LabView\\ 
\vspace*{-0.06in}
\textbf{Science Software} &
\vspace*{-0.06in}
Python packages: NumPy, SciPy and Matplotlib, GNUPlot, Scikit-learn, Astropy, SExtractor, SDSS tools \\
\vspace*{-0.06in}
\textbf{Hardware} &
\vspace*{-0.06in}
\textit{Microprocessors:} 8051, 8085, AVR and PIC, CPLDs and FPGAs, \textit{Embedded Platforms:} Arduino, RaspberryPi, \textit{standard digital logic} \\     
\end{tabular}
%
%\section{\sc Other \\Interests}
%Other than my academic interests, I like biking, long walks, swimming, socializing, cooking good food and eating it. I especially enjoy classic rock music and people who enjoy my interests.
%
%\section{\sc References} 
%\begin{tabular}{@{}p{3in}p{3in}}
%\textbf{Prof. Suyash Awate}, CSE & \textbf{Dr. Aniket Sule}, Reader \\ 
%Indian Institute of Technology, Bombay & Homi Bhabha Center for Science Education \\
%\href{mailto:suyash@cse.iitb.ac.in}{\textcolor{blue}{E--Mail}} $|$ \href{https://www.cse.iitb.ac.in/~suyash}{\textcolor{blue}{Webpage}} & \href{mailto:anikets@hbcse.tifr.res.in}{\textcolor{blue}{E--Mail}} $|$ \href{http://www.hbcse.tifr.res.in/people/academic/aniket-sule}{\textcolor{blue}{Webpage}}\\
%%\vspace*{0in}
%%\textbf{Prof. Tomislav Bujanovic}, EECS &  
%%\vspace*{0in}
%%\textbf{Prof. Bipin Rajendran}, EE\\
%%Syracuse University, NY, USA & Indian Institute of Technology, Bombay\\
%%\href{mailto:tbujanov@syr.edu}{\textcolor{blue}{E--Mail}} $|$ \href{http://eng-cs.syr.edu/about-the-college/faculty-and-staff/bujanovic}{\textcolor{blue}{Webpage}} & \href{mailto:rajendran.bipin@gmail.com}{\textcolor{blue}{E--Mail}} $|$ \href{https://sites.google.com/site/rajendranbipin/}{\textcolor{blue}{Webpage}}\\
%\end{tabular}
\end{resume}
\end{document}