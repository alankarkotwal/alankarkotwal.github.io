\documentclass[margin,line]{res}

\usepackage{hyperref}
\usepackage{amsmath}
\usepackage{textcomp}
\usepackage{color}
\oddsidemargin -.5in
\evensidemargin -.5in
\textwidth=6.0in
\itemsep=0in
\parsep=0in
% if using pdflatex:
%\setlength{\pdfpagewidth}{\paperwidth}
%\setlength{\pdfpageheight}{\paperheight} 

\newenvironment{list1}{
  \begin{list}{\ding{113}}{%
      \setlength{\itemsep}{0in}
      \setlength{\parsep}{0in} \setlength{\parskip}{0in}
      \setlength{\topsep}{0in} \setlength{\partopsep}{0in} 
      \setlength{\leftmargin}{0.17in}}}{\end{list}}
\newenvironment{list2}{
  \begin{list}{$\bullet$}{%
      \setlength{\itemsep}{0in}
      \setlength{\parsep}{0in} \setlength{\parskip}{0in}
      \setlength{\topsep}{0in} \setlength{\partopsep}{0in} 
      \setlength{\leftmargin}{0.2in}}}{\end{list}}

\tolerance=1
\emergencystretch=\maxdimen
\hyphenpenalty=10000
\hbadness=10000

\pagenumbering{gobble}

\begin{document}

\name{\textbf{\huge{Alankar Kotwal}} \vspace*{.05in} \newline  {\sc Senior Undergraduate}\vspace*{.1in}}

\begin{resume}
\section{\sc Contact Information}
\vspace{.05in}
\begin{tabular}{@{}p{2.9in}p{.5in}p{3in}}
Department of Electrical Engineering & \multicolumn{1}{r}{\it Phone:}  &(+91) 996 967 8123 \\            
Indian Institute of Technology Bombay &\multicolumn{1}{r}{\it E--Mail:}& \href{mailto:alankar.kotwal@iitb.ac.in}{\textcolor{blue}{alankar.kotwal@iitb.ac.in}} \\ 
281, Hostel 09, IIT Bombay & & \href{mailto:alankarkotwal13@gmail.com}{\textcolor{blue}{alankarkotwal13@gmail.com}} \\ 
Powai, Mumbai, India 400 076 & \multicolumn{1}{r}{\it Webpage:} &\href{http://alankarkotwal.github.io/}{\textcolor{blue}{alankarkotwal.github.io}} \\     
\end{tabular}

\section{\sc Research Interests}
I am passionate about Computer and Medical Vision, Machine Learning, Estimation Theory, Astrophysics and Cosmology. I enjoy learning about and experimenting with Robotics, Computer Networks and Security, Computer Graphics and applications of these fields in one another.

\section{\sc Education}
{\bf \href{http://www.iitb.ac.in/}{\textcolor{blue}{Indian Institute of Technology Bombay}}}, Mumbai, India \hfill {\it July 2012 -- Present} \\
\vspace*{-.1in}
\begin{list1}
\item[] Dual Degree, Bachelor \& Master of Technology, \href{http://www.ee.iitb.ac.in/}{\textcolor{blue}{Department of Electrical Engineering}}
\item[] Specialization: {\em Communication and Signal Processing}
\begin{list2}
\vspace*{.05in}
\item \textbf{Major CGPA:}  \ 8.78/10 (\hyperlink{crselst}{\textcolor{blue}{Detailed List of Courses}})
\item \textbf{Minor Degree:}  Department of \href{www.cse.iitb.ac.in}{\textcolor{blue}{Computer Science \& Engineering}}
\end{list2}
\end{list1}

\section{\sc Research Internships} 

{\bf  \href{http://theairlab.org/}{\textcolor{blue}{The AIR Lab}}, \href{http://www.cmu.edu/}{\textcolor{blue}{Carnegie Mellon University}} \href{http://ri.cmu.edu/}{\textcolor{blue}{Robotics Institute}}} \\
{\em Guide: \href{http://www.ri.cmu.edu/person.html?person_id=1397}{\textcolor{blue}{Prof. Sebastian Scherer}} \& \href{http://www.ri.cmu.edu/person.html?person_id=2128}{\textcolor{blue} {Stephen Nuske}}}\hfill\textit{Summer 2015} \\
\vspace*{-.13in}
\begin{list1}
\item[]\textbf{Stereo Odometry From A Downward-Facing Stereo Camera On A Vehicle} \\
Fast and accurate stereo odometry is a pre-requisite for many robotics applications like localisation, path planning and navigation. For aerial vehicles like quadcopters, a good way to do odometry is to use a ground-facing camera and track the motion of the (relatively featureless) ground. This has traditionally been done with sensors like the Pixhawk PX4FLOW, which uses a single camera doing correlation-based tracking along with a sonar for odometry. This has several disadvantages, like small camera field of view (meaning small maximum allowed speeds for accurate tracking), bad sonar readings at low range (especially during take-off), requirement of an inertial unit for angle measurement and height-dependent camera focus. We aimed to replace the PX4FLOW with a small-baseline stereo camera for the same purpose. Assuming that most of the field of view lies on a plane parallel to the sensors, the height of the vehicle is obtained from a robust estimate of the horizontal disparity between rectified stereo pairs. Alternatively, height, pitch and roll are jointly estimated using a \textit{robust gradient-descent homography fit} between rectified stereo pairs. Similar, \textit{rigid tracking across frames} is then used to measure position. We obtained better depth estimates, better maximum speeds and comparable accuracy without an inertial unit as compared to the PX4FLOW.
\end{list1}

{\bf \href{http://lcdm.astro.illinois.edu/}{\textcolor{blue} {Laboratory for Cosmological Data Mining}}, \href{http://www.illinois.edu/}{\textcolor{blue}{University of Illinois, Urbana -- Champaign}}} \\
{\em Guide: \href{hhttp://www.astro.illinois.edu/people/bigdog}{\textcolor{blue}{Prof. Robert Brunner}}, under \href{https://www.google-melange.com/gsoc/homepage/google/gsoc2014}{\textcolor{blue} {Google Summer of Code}}} \hfill {\it Summer 2014} \\
\vspace*{-.13in}
\begin{list1}
\item[]\textbf{A Pixel-Level Machine Learning Method for Calculating Source Redshifts} \\
Distances in Astrophysics have traditionally been measured using a variety of techniques, spectrometry prominent among them. The basic idea in spectrometry is, given a source with a measurable spectrum, features in the spectrum (like emission or absorption lines) can be fit with known lines to obtain the source's redshift, which is a measure of distance at cosmologically significant distances. However, there exist sources which are either very far or very dim, so we do not get enough flux from them to measure their spectrum. Broad-band energies from these sources, as an approximation to the entire spectrum, are used as features for a machine learning algorithm to calculate redshifts for these sources, or alternatively classify them. Unlike previous attempts, we \textit{calculate features pixel-wise} instead of integrating over entire source area, giving potential benefits like \textit{source de-blending and better background separation}. The redshift calculation and source classification from the method are reasonably accurate.
\end{list1}

%\vspace{-.1cm}
\newpage
\section{\sc Research Projects}

{\bf A New Bayesian Framework For Laparoscopic Image Dehazing and Denoising} \\
{\em Guide: \href{https://www.cse.iitb.ac.in/~suyash}{\textcolor{blue}{Prof. Suyash Awate}}, CSE, IITB} \hfill {\it January 2015 -- Present} \\
\vspace*{-.13in}
\begin{list1}
\item[]
Laparoscopic images in minimally invasive surgery get corrupted by surgical smoke and noise. This degrades the quality of the surgery and the results of subsequent processing for, say, segmentation and tracking. Algorithms for desmoking and denoising laparoscopic images seem to be missing in the medical vision literature. We formulated the problem of \textit{joint desmoking and denoising} of laparoscopic images as a \textit{Bayesian inference} problem. This formulation relies on a novel \textit{probabilistic graphical model} of images, which includes a \textit{Markov Random Field (MRF) formulation for color-contrast} and another \textit{MRF for smoothness} on the uncorrupted color image as well as the transmission-map image that indicates color attenuation due to smoke. The results on simulated and real-world laparoscopic images, including clinical expert evaluation, shows the advantages of the proposed method over the state of the art. The results have been submitted to the \href{http://biomedicalimaging.org/2016/}{\textcolor{blue} {International Symposium on Biomedical Imaging}} for publication.
\end{list1}

{\bf The IITB Mars Rover Project}
%\\
%{\em Guide: \href{http://www.aero.iitb.ac.in/~pjguru/}{\textcolor{blue}{Prof. PJ Guruprasad}}, Aerospace, IITB}
\hfill {\it May 2013 -- May 2015} \\
\vspace*{-.13in}
\begin{list1}
\item[]
The IITB Mars Rover project is a student initiative at IIT Bombay to build a prototype Mars rover capable of a number of tasks required for extra-terrestrial robotics. The aim of the project is to participate in the \href{urc.marssociety.org/}{\textcolor{blue} {University Rover Challenge}}. We designed and developed a rover with a \textit{rocker-bogie suspension and novel air-filled beach tyres}. As a part of the electrical and on-board computing team, we designed power, logic and commnuication circuits for on-board control. We developed localisation and autonomous navigation. The role of \textit{stereo vision and image processing} in making rover operations autonomous was explored. We got the opportunity to participate in a simulated Martian expedition and test the Rover's capabilities in the Australian outback, at the \href{http://marssociety.org.au/article/arkaroola-mars-robot-challenge-spaceward-bound-expedition}{\textcolor{blue} {Arkaroola Mars Robot Challenge}} (expedition details published in \href{http://www.nssa.com.au/14asrc/14ASRC-proceedings.zip}{\textcolor{blue} {here}}) and at the Mars Society's \href{http://mdrs.marssociety.org/}{\textcolor{blue} {Mars Desert Research Station}}.
\end{list1}

\section{\sc Course \\Projects}
{\bf Improved Methods for Compressed Sensing Recovery} \hfill {\it CS709: Convex Optimization} \\
{\em Guide: \href{https://www.cse.iitb.ac.in/~ganesh/}{\textcolor{blue}{Prof. Ganesh Ramakrishnan}}, CSE, IITB \hfill Autumn 2015-16} \\
\vspace*{-.15in}
\begin{list1}
\item[] Using a series of convex approximations to the compressed sensing recovery problem, we reconstructed good and near-exact versions of the Barbara image at compression levels of as low as $0.1$ and $0.2$ respectively. We also proved convergence of the algorithm to the exact solution.
\end{list1}

\vspace*{-0.1in}

{\bf Part-of-Speech Tagging} \hfill \textit{EE638: Estimation and Identification} \\
{\em Guide: \href{https://www.ee.iitb.ac.in/course/~ee638/Navin}{\textcolor{blue}{Prof. Navin Khaneja}}, EE, IITB \hfill Autumn 2015-16} \\
\vspace*{-.15in}
\begin{list1}
\item[] We implemented Hidden Markov Model-based part-of-speech tagging with support for unknown words. An error rate of around 5\% and capabilities of the system to discern context were observed.
\end{list1}

\vspace*{-0.1in}

{\bf Laparoscopic Image Dehazing} \hfill \textit{CS736: Medical Image Processing} \\
{\em Guide: \href{https://www.cse.iitb.ac.in/~suyash}{\textcolor{blue}{Prof. Suyash Awate}}, CSE, IITB \hfill Spring 2014-15} \\
\vspace*{-.15in}
\begin{list1}
\item[] We applied the Dark Channel Prior method for landscape image dehazing to laparoscopic images. In order to make the process real-time, we replaced refining the transmission map with a differential equation with guided filtering and got good results.
\end{list1}

\vspace*{-0.1in}

{\bf Stereo Odometry Via Point Cloud Registration} \hfill \textit{CS763: Computer Vision} \\
{\em Guide: \href{https://www.cse.iitb.ac.in/~ajitvr}{\textcolor{blue}{Prof. Ajit Rajwade}}, CSE, IITB \hfill Spring 2014-15} \\
\vspace*{-.15in}
\begin{list1}
\item[] We explored kernel density correlation as a method for registering pointclouds. We performed this maximisation with gradient-ascent and coherent point drift with PCL in C++ and observed good convergence behaviour for small displacements and rotations.
\end{list1}

\vspace*{-0.1in}

{\bf Gravitational Lens Separation With PCA} \hfill \textit{CS663: Digital Image Processing} \\
{\em Guide: \href{https://www.cse.iitb.ac.in/~suyash}{\textcolor{blue}{Prof. Suyash Awate}} and \href{https://www.cse.iitb.ac.in/~ajitvr}{\textcolor{blue}{Prof. Ajit Rajwade}}, CSE, IITB \hfill Autumn 2014-15} \\
\vspace*{-.15in}
\begin{list1}
\item[] Gravitationally lensed images of galaxies have rare arc-like artifacts that can be used to calculate the mass of the lens. Detection and source subtraction are essential preprocessing steps for this. We used Anscombe denoising followed by PCA to build a basis for galaxy images and used the top few eigengalaxies to reconstruct and subtract sources to detect arc-like artifacts.
\end{list1}

\vspace*{-0.1in}

{\bf Processor Design} \hfill \textit{EE309: Microprocessors} \\
{\em Guide: \href{https://www.ee.iitb.ac.in/~viren/}{\textcolor{blue}{Prof. Virendra Singh}}, EE, IITB \hfill Autumn 2014-15} \\
\vspace*{-.15in}
\begin{list1}
\item[] We designed, simulated and implemented a multi-cycle RISC processor following the LC-3b ISA. The implementation was done on a DE0-Nano board from Terasic. Following this, we designed and simulated a pipelined RISC processor using the Little Computer Architecture.
\end{list1}

\vspace*{-0.1in}

{\bf A PD Temperature Controller on an Logic Device} \hfill \textit{EE224: Digital Systems Lab} \\
{\em Guide: \href{https://www.ee.iitb.ac.in/wiki/faculty/jayanta}{\textcolor{blue}{Prof. Jayanta Mukherjee}}, EE, IITB \hfill Spring 2013-14} \\
\vspace*{-.15in}
\begin{list1}
\item[] We designed, simulated and implemented a proportional-derivative temperature controller with a Peltier plate used as a heating/cooling element and an LM35 temperature sensor. We observed quick temperature rise and stable steady-state with best-tuned parameters for the controller.
\end{list1}

\vspace*{-0.1in}

\section{\sc Astrophysics \\Projects}

{\bf An X-Ray Study of Black Hole Candidate X Norma X-1} \hfill \textit{\href{http://nius.hbcse.tifr.res.in/}{\textcolor{blue} {NIUS, Astronomy}}} \\
{\em Guide: \href{http://cbs.ac.in/people/visiting-scientists/manojendu-choudhury}{\textcolor{blue}{Prof. Manojendu Choudhury}}, \href{http://cbs.ac.in/}{\textcolor{blue} {Center for Basic Sciences}}, University of Mumbai \hfill Dec 2013}
\vspace*{-.15in}
\begin{list1}
\item[] We analyzed spectral data from the low-mass X-Ray Binary 4U 1630-47 (X Nor X-1) for the period between MJD 53829.3638299 and MJD 53950.93011. This period corresponds to an outburst in the source. The data is taken from the RXTE GOF archives. We extracted 3-30 keV spectra and fit them with a theoretical model which, in addition to the main radiation mechanism, takes into account interstellar extinction along the line of sight. The model for the main radiation mechanism consists of thermal emission from a geometrically thin and optically thick disk, and non-thermal radiation modeled by a power-law, presumably from a high energy Comptonizing cloud located inside the truncated disk, as well as emission of Compton radiation from the disk. We obtained best fit values of various parameters like the internal radius of the accretion disk, the internal temperature, flux in various bandpasses, relative magnitudes of the non-thermal Compton component and the thermal blackbody component.
\end{list1}

\vspace*{-0.1in}

{\bf Estimation of Photometric Redshifts Using Machine Learning} \hfill \textit{\href{http://nius.hbcse.tifr.res.in/}{\textcolor{blue} {NIUS, Astronomy}}} \\
{\em Guide: \href{http://www.iucaa.ernet.in/~nspp/}{\textcolor{blue}{Prof. Ninan Sajeeth Philip}}, \href{http://www.iucaa.ernet.in/}{\textcolor{blue} {IUCAA}}, Pune \hfill Dec 2012} \\
\vspace*{-.15in}
\begin{list1}
\item[] Broad-band energies in color filters are used as inputs to a machine learning algorithm for determination of redshifts for objects whose spectra cannot be measured accurately. Here, we trained a two-layer neural network to calculate photometric redshifts for such objects and obtained good results. Often, data that we get from sources whose redshift is known from other methoda is not enough to train learning algorithms for prediction with reasonable accuracy. Hence, we worked on data generation from available data. We redshifted available spectra to determine how the object with a given spectrum would look like at a range of higher redshifts. From these redshifted spectra we extracted energies in color filters and used this (nearly 10-fold) expanded dataset to train the learning algorithm. We achieved good predictions for test data and observed clustering of galaxy colors as a function of increasing redshift.
\end{list1}

\section{\sc Achievements \\and Awards}
\begin{list2}
\item[\strut\hspace{0.5cm}\textbf{Olympiads and Competitive Exams}]
\item Represented India at the $6^{th}$ International Olympiad on Astronomy and Astrophysics, Brazil, 2012. Won a Gold Medal with International Rank 4 and a special prize for Best Data Analysis
\item Represented India at the $5^{th}$ International Earth Sciences Olympiad, Italy, 2011. Won a Bronze Medal and prizes for best performance in the Hydrosphere section and the team presentation
\item  Secured All India Rank (AIR) $105$ in \href{https://en.wikipedia.org/wiki/Indian_Institute_of_Technology_Joint_Entrance_Examination}{\textcolor{blue}{IIT-JEE}} amongst $1.1$ million candidates
\vspace{0.05in}
\item[\strut\hspace{0.5cm}\textbf{Scholarships}]
\item Awarded \href{http://www.kvpy.iisc.ernet.in/main/index.htm}{\textcolor{blue}{KVPY Scholarship}} $2011$ by Dept. of Science and Technology, Govt. of India
\item Awarded \href{http://www.ncert.nic.in/programmes/talent_exam/index_talent.html}{\textcolor{blue}{NTSE Scholarship}} $2008$ by NCERT, Govt. of India
\item[\strut\hspace{0.5cm}\textbf{Competitions}]
\item Secured IIT Bombay the second position by putting on board 72 Messier objects including the entire Virgo cluster of galaxies in the Inter-IIT Messier Marathon, 2013
\item Won the Astronomy Quiz conducted by the Astronomy Club, IIT Bombay, 2012
\end{list2}

\section{\sc Talks and \\Seminars}

{\bf Template-Based Stereo Odometry} \hfill {\em Invited Talk} \\
{\em \href{http://theairlab.org/}{\textcolor{blue}{The AIR Lab}}, \href{http://www.cmu.edu/}{\textcolor{blue}{Carnegie Mellon University}} \hfill July 2015} \\
\vspace*{-.15in}
\begin{list1}
\item[] Here, I presented results from my 2015 summer internship to my group at Carnegie Mellon University. The talk included a detailed description of the method used, comparisons of the results with ground-truth and stress-tests on the method. The presentation may be found \href{http://alankarkotwal.github.io/intern_presentation.pptx}{\textcolor{blue} {here}}.
\end{list1}

{\bf \href{http://www.stab-iitb.org/krittika/the-cosmic-ladder-distance}{\textcolor{blue} {The Cosmic Distance Ladder}}} \hfill {\em Invited Talk} \\
{\em \href{http://www.stab-iitb.org/krittika/}{\textcolor{blue} {Krittika -- The Astronomy Club, IIT Bombay}} \hfill Dec 2012} \\
\vspace*{-.15in}
\begin{list1}
\item[] This open-to-all talk is a journey climbing the Cosmic Distance Ladder, which is a sequence of steps, each building on the previous step's results, for calculating distances in the universe. We begin with solar system distances, and end at huge distances where the only real option is to use photometric redshifts. This talk also presents results from my Google Summer of Code project. The presentation may be found \href{http://alankarkotwal.github.io/CosmicDistanceLadder.pptx}{\textcolor{blue} {here}}.
\end{list1}


\section{\sc Mentoring Experience}
\textbf{Teaching Assistant}
\begin{list1}
\item[]CS663: Digital Image Processing \hspace{0.5cm} \href{https://www.cse.iitb.ac.in/~suyash}{\textcolor{blue}{Prof. S. Awate}} and \href{https://www.cse.iitb.ac.in/~ajitvr}{\textcolor{blue}{Prof. A. Rajwade}} \hfill{\textit{Autumn 2015-16}}
\end{list1}

\vspace*{-0.1in}

\textbf{Resource Person, Indian Astronomy Olympiad Programme} \hfill \textit{May 2013, May 2014} \\
\vspace*{-.15in}
\begin{list1}
\item[] Selected twice as a resource person for the Indian Astronomy Olympiad Camp, for their selection to the international Astronomy Olympiads. Involved in mentoring students ranging from the $9^{th}$ to the $12^{th}$ grades in Astronomy, and in setting up challenging questions and evaluating students.
\end{list1}

\vspace*{-0.1in}

\textbf{Technical Mentor}\hfill \textit{April 2013 -- March 2014} \\
\vspace*{-.15in}
\begin{list1}
\item[] Mentored $1^{st}$ year students for Robotics Competitions and Institute Technical Summer Projects.
\end{list1}

\section{\sc Relevant \\Coursework} 
\begin{list1}
\item[\strut\hspace{0.5cm}\hypertarget{crselst}{\textbf{Computer Sciences and Engineering}}]
\item[]\textit{Computer Graphics, Computer Vision, Algorithms for Medical Image Processing, Machine Learning, Convex Optimisation, Digital Image Processing, Design and Analysis of Algorithms, Data Structures and Algorithms, Discrete Mathematics}
\item[\strut\hspace{0.5cm}\textbf{Electrical Engineering}]
\vspace{0.05in}
\item[]\textit{Estimation and Identification, Speech Processing, Digital Signal Processing, Controls, Probability and Random Processes, Digital Communication, Communication Systems, Microprocessors, Signals and Systems, Digital and Analog Systems, Electronic Devices and Circuits, Network Theory}
\item[\strut\hspace{0.5cm}\textbf{Physics and Mathematics}]
\vspace{0.05in}
\item[]\textit{The General Theory of Relativity, Electromagnetic Waves, Electricity and Magnetism, Classical Mechanics, Differential Equations, Linear Algebra, Complex Analysis, Calculus}
\end{list1}

\section{\sc Technical \\Skills} 
\begin{tabular}{@{}p{1.3in}p{4.3in}}
\textbf{Programming} & C/C++, Python, Shell Scripting, Matlab, SQL, HTML, PHP, \LaTeX \\  
\vspace*{-0.06in}
\textbf{Software Packages} & 
\vspace*{-0.06in}
ROS/Gazebo, OpenCV, The Point Cloud Library, SPICE Circuit Simulation, EAGLE PCB Design, SolidWorks, AutoCAD, LabView\\ 
\vspace*{-0.06in}
\textbf{Science Software} &
\vspace*{-0.06in}
Python packages: NumPy, SciPy and Matplotlib, GNUPlot, Scikit-learn, Astropy, SExtractor, SDSS tools \\
\vspace*{-0.06in}
\textbf{Hardware} &
\vspace*{-0.06in}
\textit{Microprocessor Architectures:} 8051, 8085, AVR and PIC, CPLDs and FPGAs, \textit{Embedded Platforms:} Arduino, RaspberryPi, \textit{standard digital logic families} \\     
\end{tabular}

\section{\sc Other \\Interests}
Other than my academic interests, I like biking, long walks, swimming, socializing, cooking good food and eating it. I especially enjoy classic rock music and people who enjoy my interests.

\section{\sc References} 
\begin{tabular}{@{}p{3in}p{3in}}
\textbf{Prof. Suyash Awate}, CSE & \textbf{Dr. Aniket Sule}, Reader \\ 
Indian Institute of Technology, Bombay & Homi Bhabha Center for Science Education \\
\href{mailto:suyash@cse.iitb.ac.in}{\textcolor{blue}{E--Mail}} $|$ \href{https://www.cse.iitb.ac.in/~suyash}{\textcolor{blue}{Webpage}} & \href{mailto:anikets@hbcse.tifr.res.in}{\textcolor{blue}{E--Mail}} $|$ \href{http://www.hbcse.tifr.res.in/people/academic/aniket-sule}{\textcolor{blue}{Webpage}}\\
%\vspace*{0in}
%\textbf{Prof. Tomislav Bujanovic}, EECS &  
%\vspace*{0in}
%\textbf{Prof. Bipin Rajendran}, EE\\
%Syracuse University, NY, USA & Indian Institute of Technology, Bombay\\
%\href{mailto:tbujanov@syr.edu}{\textcolor{blue}{E--Mail}} $|$ \href{http://eng-cs.syr.edu/about-the-college/faculty-and-staff/bujanovic}{\textcolor{blue}{Webpage}} & \href{mailto:rajendran.bipin@gmail.com}{\textcolor{blue}{E--Mail}} $|$ \href{https://sites.google.com/site/rajendranbipin/}{\textcolor{blue}{Webpage}}\\
\end{tabular}
\end{resume}
\end{document}