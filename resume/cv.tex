\documentclass[margin,line]{res}

\usepackage{hyperref}
\usepackage{amsmath}
\usepackage{textcomp}
\usepackage{color}
\usepackage{lettrine}
\oddsidemargin = -.5in
\evensidemargin = -.5in

\textwidth = 6.0in
\itemsep=0in
\parsep=0in
% if using pdflatex:
%\setlength{\pdfpagewidth}{\paperwidth}
%\setlength{\pdfpageheight}{\paperheight} 

\newenvironment{list1}{
  \begin{list}{\ding{113}}{%
      \setlength{\itemsep}{0in}
      \setlength{\parsep}{0in} \setlength{\parskip}{0in}
      \setlength{\topsep}{0in} \setlength{\partopsep}{0in} 
      \setlength{\leftmargin}{0.17in}}}{\end{list}}
\newenvironment{list2}{
  \begin{list}{$\bullet$}{%
      \setlength{\itemsep}{0in}
      \setlength{\parsep}{0in} \setlength{\parskip}{0in}
      \setlength{\topsep}{0in} \setlength{\partopsep}{0in} 
      \setlength{\leftmargin}{0.2in}}}{\end{list}}

\tolerance=1
\emergencystretch=\maxdimen
\hyphenpenalty=10000
\hbadness=10000

\pagenumbering{gobble}

\begin{document}

%\name{\textbf{\huge{Alankar Kotwal}} \vspace*{.05in} \newline  {\sc Senior Undergraduate}\vspace*{.1in}}
\name{\textbf{\huge{Alankar Kotwal}} \vspace*{.1in} }

\begin{resume}
\section{\sc Contact Information}
\vspace{.05in}
\begin{tabular}{@{}p{2.9in}p{.5in}p{3in}}
Department of Electrical Engineering & \multicolumn{1}{r}{\it Phone:}  &(+91) 996 967 8123 \\            
Indian Institute of Technology Bombay &\multicolumn{1}{r}{\it E--Mail:}& \href{mailto:alankar.kotwal@iitb.ac.in}{\textcolor{blue}{alankar.kotwal@iitb.ac.in}} \\ 
\#281, Hostel 09, IIT Bombay & & \href{mailto:alankarkotwal13@gmail.com}{\textcolor{blue}{alankarkotwal13@gmail.com}} \\ 
Powai, Mumbai, India 400 076 & \multicolumn{1}{r}{\it Webpage:} &\href{http://alankarkotwal.github.io/}{\textcolor{blue}{alankarkotwal.github.io}} \\     
\end{tabular}

\section{\sc Research Interests}
\lettrine[lines=2]{I}{} am passionate about Computer and Medical Vision (visual recognition, imaging modalities, inverse problems, computational photography), Compressed Sensing (theoretical guarantees, sensing matrix optimization), Machine Learning (graphical models, language processing), Optimization, Astrophysics (stellar structure and evolution) and Cosmology (the $\Lambda$CDM model and constituent resolution). I enjoy learning about and experimenting with Robotics (navigation systems, extraterrestrial robotics), Computer Networks and Security, Graphics and applications of these fields in one another.

\section{\sc Education}
{\bf \href{http://www.iitb.ac.in/}{\textcolor{blue}{Indian Institute of Technology Bombay}}}, Mumbai, India \hfill {\it July 2012 -- Present} \\
\vspace*{-.1in}
\begin{list1}
\item[] Final Year, Dual Degree (Bachelor \& Master of Technology), Department of \href{http://www.ee.iitb.ac.in/}{\textcolor{blue}{Electrical Engineering}}
\item[] Specialization: {\em Communication and Signal Processing}
\begin{list2}
\vspace*{.05in}
\item \textbf{Major CGPA:}  \ 8.92/10 (\hyperlink{crselst}{\textcolor{blue}{Detailed List of Courses}})
\item \textbf{Minor Degree:}  Department of \href{http://www.cse.iitb.ac.in/}{\textcolor{blue}{Computer Science \& Engineering}}
\end{list2}
\end{list1}

{\bf Ratanbai Walbai Junior Science College}, Mumbai, India \hfill {\it July 2010 -- April 2012} \\
\vspace*{-.1in}
\begin{list1}
\item[] Intermediate Education: Physics, Chemistry and Maths
\item[] Specialization: {\em Electrical Maintenance}
\begin{list2}
\vspace*{.05in}
\item \textbf{Major CGPA:}  \ 9.38/10
\end{list2}
\end{list1}

{\bf Saraswati Vidyalaya High School}, Thane, India \hfill {\it July 2000 -- April 2010} \\
\vspace*{-.1in}
\begin{list1}
\item[] Matriculation
\begin{list2}
\vspace*{.05in}
\item \textbf{Major CGPA:}  \ 9.53/10
\end{list2}
\end{list1}

\section{\sc Publications}
\begin{list2}
\item Kotwal, A., Bhalodia, R., Awate, S., {\em Joint Desmoking and Denoising of Laparoscopy Images} (oral), Proc. of the \href{http://biomedicalimaging.org/2016/}{\textcolor{blue} {International Symposium on Biomedical Imaging}}, 2016. Paper \href{http://alankarkotwal.github.io/pubs/lap-dehazing.pdf}{\textcolor{blue} {here}}.
\item Clarke, J. D. A., Held, J. M., Dahl, A. {\em et al.}, {\em Field Robotics, Astrobiology and Mars Analogue Research on the Arkaroola Mars Robot Challenge Expedition}, Proc. of the \href{http://www.nssa.com.au/14asrc/14ASRC-proceedings.zip}{\textcolor{blue} {14th Australian Space Research Conference}}, 2014. Paper \href{http://alankarkotwal.github.io/pubs/Arkaroola.pdf}{\textcolor{blue} {here}}.
\end{list2}

\section{\sc Research Internships} 

{\bf  \href{http://theairlab.org/}{\textcolor{blue}{The AIR Lab}}, \href{http://www.cmu.edu/}{\textcolor{blue}{Carnegie Mellon University}} \href{http://ri.cmu.edu/}{\textcolor{blue}{Robotics Institute}}} \\
{\em Guide: \href{http://www.ri.cmu.edu/person.html?person_id=1397}{\textcolor{blue}{Prof. Sebastian Scherer}} \& \href{http://www.ri.cmu.edu/person.html?person_id=2128}{\textcolor{blue} {Stephen Nuske}}} \hfill {\it Summer 2015} \\
\vspace*{-.13in}
\begin{list1}
\item[]\textbf{Stereo Odometry From A Downward-Facing Stereo Camera On An Aerial Vehicle} \\
For aerial vehicles, odometry is often done by using sensors like the Pixhawk PX4FLOW, which use a single camera doing correlation-based tracking with a sonar for odometry. This has many disadvantages, like small camera field of view (small maximum speeds), bad sonar readings at low range (during take-off), requirement of an inertial unit for angle measurement and height-dependent camera focus. We aimed to replace this with a small-baseline stereo camera. With the field of view parallel to the baseline, the height of the vehicle is obtained from a robust estimate of horizontal disparity. Alternatively, height, pitch and roll are jointly estimated using a robust gradient-descent homography fit between stereo pairs. Similar, rigid tracking across frames is then used to measure position. We obtained better height estimates, maximum speeds and comparable accuracy without an inertial unit as compared to the PX4FLOW. Code \href{https://github.com/alankarkotwal/ground_odom}{\textcolor{blue} {here}}.
\end{list1}

{\bf \href{http://lcdm.astro.illinois.edu/}{\textcolor{blue} {Laboratory for Cosmological Data Mining}}, \href{http://www.illinois.edu/}{\textcolor{blue}{University of Illinois, Urbana -- Champaign}}} \\
{\em Guide: \href{http://www.astro.illinois.edu/people/bigdog}{\textcolor{blue}{Prof. Robert Brunner}}, under \href{https://www.google-melange.com/gsoc/homepage/google/gsoc2014}{\textcolor{blue} {Google Summer of Code}}} \hfill {\it Summer 2014} \\
\vspace*{-.13in}
\begin{list1}
\item[]\textbf{A Pixel-Level Machine Learning Method for Calculating Source Redshifts} \\
Spectrometry is a prominent distance measurement technique in Astrophysics. Here, features in the spectrum (like emission or absorption lines) can be fit with known lines to obtain redshift, which is a measure of distance at cosmologically significant distances. However, there exist sources which are either very far or very dim, so we do not get enough flux from them to measure their spectrum. Broad-band energies from these sources, as an approximation to the entire spectrum, are used as features for a machine learning algorithm to calculate redshifts, or alternatively classify them. Unlike previous work, we calculate features pixel-wise instead of integrating over entire source area, giving benefits like source de-blending and better background separation. Redshift calculation and source classification from the method are reasonably accurate. Code \href{https://github.com/alankarkotwal/image-photo-z/}{\textcolor{blue} {here}}.
\end{list1}

{\bf Srujana -- Center for Innovation, \href{http://www.lvpei.org/}{\textcolor{blue}{L. V. Prasad Eye Institute}}} \\
{\em Guide: \href{http://www.lvpei.org/our-team/our-team-ashutosh.php}{\textcolor{blue}{Ashutosh Richhariya}}, Ophthalmic Biophysics, LVPEI} \hfill {\it Winter 2014} \\
\vspace*{-.13in}
\begin{list1}
\item[]\textbf{Super--Resolution with Fourier Ptychographic Microscopy} \\
The maximum number of independent samples that an imaging setup can extract from the imaged scene is dictated by the limits set by Fourier optics: if the imaged scene is wide--field, there are limits on how much one can zoom in. However, wide--field, high--resolution images are generally desirable in microscopy, pathology and eye imaging. Traditional methods to achieve this involve mechanical adjustment and require precise control and actuation. Fourier Ptychographic Microscopy was introduced in 2013 as a computational tool to work around this. The idea is to shift, in the Fourier domain, high--frequency information to low frequencies (where the system's optics do not filter it), sense it, and shift it back while reconstructing. We worked on understanding and implementing this method for microscopy slides, and analyzed possible extensions to imaging reflective surfaces like the eye.
Code \href{https://github.com/alankarkotwal/lvp-imaging/tree/parallel}{\textcolor{blue} {here}}.
\end{list1}

\vspace{-.1cm}

\section{\sc Research Projects}

{\bf A Bayesian Framework For Laparoscopic Image Dehazing and Denoising} \\
{\em Guide: \href{https://www.cse.iitb.ac.in/~suyash}{\textcolor{blue}{Prof. Suyash Awate}}, CSE, IITB} \hfill {\it January 2015 -- Present} \\
\vspace*{-.13in}
\begin{list1}
\item[]
Laparoscopic images in minimally invasive surgery get corrupted by surgical smoke and noise. This degrades the quality of the surgery and the results of further processing for, say, segmentation and tracking. Algorithms for desmoking and denoising laparoscopic images seem to be missing in the medical vision literature. We formulated the problem of joint desmoking and denoising of laparoscopic images as a Bayesian inference problem. This formulation relies on a novel probabilistic graphical model of images, which includes a Markov Random Field (MRF) formulation for color-contrast and another MRF for smoothness on the uncorrupted color image as well as the transmission-map image that indicates color attenuation due to smoke. The results on simulated and real-world laparoscopic images, with clinical expert evaluation, shows the advantages of our method over the state of the art. Code \href{https://github.com/alankarkotwal/lap-dehazing}{\textcolor{blue} {here}}.
\end{list1}

{\bf Coded Source Separation for Compressed Video Recovery} \hfill \textit{Dual Degree Thesis} \\
{\em Guide: \href{https://www.cse.iitb.ac.in/~ajitvr}{\textcolor{blue}{Prof. Ajit Rajwade}}, CSE \& \href{https://www.ee.iitb.ac.in/wiki/faculty/rajbabu}{\textcolor{blue}{Prof. V. Rajbabu}}, EE, IITB} \hfill {\it December 2015 -- Present} \\
\vspace*{-.13in}
\begin{list1}
\item[]
Recent efforts to apply the principles of compressed sensing to video data involve combining frames into coded snapshots while sensing and separating them with an over--complete dictionary. This works well, but needs a dictionary at the same frame--rate and time--smoothness as the video. We try relaxing this constraint using a source--separation approach to this problem where precise error bounds on recovery have been derived. Basis pursuit recovery with Gaussian--random sensing matrices gives excellent results with no ghosting for both similar and radically different images. Unfortunately, the more realizable non--negative sensing matrices don't work as well, because they do not have the nice incoherence properties of Gaussian--random matrices. We aim to design such sensing matrices with low mutual coherence, making them ideal for compressed video. We also aim to design matrices that when rotated circularly, still have low mutual coherence so that they can be tiled for patch-wise reconstruction. Code \href{https://bitbucket.org/alankarkotwal/coded-sourcesep}{\textcolor{blue} {here}}.
\end{list1}

{\bf The IITB Mars Rover Project}
%\\
%{\em Guide: \href{http://www.aero.iitb.ac.in/~pjguru/}{\textcolor{blue}{Prof. PJ Guruprasad}}, Aerospace, IITB}
\hfill {\it May 2013 -- Present} \\
\vspace*{-.13in}
\begin{list1}
\item[]
The IITB Mars Rover project is a student initiative at IIT Bombay to build a prototype Mars rover capable of extra-terrestrial robotics and to participate in the \href{http://urc.marssociety.org/}{\textcolor{blue} {University Rover Challenge}} at the Mars Society's \href{http://mdrs.marssociety.org/}{\textcolor{blue} {Mars Desert Research Station}}, Utah. The mechanical subsystem designed and developed a rover with a rocker-bogie suspension and novel air-filled beach tires. The electrical and software team designed power, logic and communication circuits for on-board control. Currently, localization and autonomous navigation are being developed. The role of machine vision for automating rover operations is being explored. One of the design goals for the future is to develop the capability to help astronauts on space missions. We participated in a simulated Martian expedition and tested Rover capabilities in the harsh conditions of the Australian outback, at the \href{http://marssociety.org.au/article/arkaroola-mars-robot-challenge-spaceward-bound-expedition}{\textcolor{blue} {Arkaroola Mars Robot Challenge}}. We participated in a series of exercises in Mars Operations Research, involving test extra--vehicular activities in space--suits (a sample collection task and a rover guidance task) conducted by \href{https://saberastro.com/}{\textcolor{blue} {Saber Astronautics}}. Details and pictures \href{http://alankarkotwal.github.io/#projects}{\textcolor{blue} {here}}.
\end{list1}

\section{\sc Course \\Projects}
{\bf Improved Methods for Compressed Sensing Recovery} \hfill {\it CS709: Convex Optimization} \\
{\em Guide: \href{https://www.cse.iitb.ac.in/~ganesh/}{\textcolor{blue}{Prof. Ganesh Ramakrishnan}}, CSE, IITB \hfill Autumn 2015-16} \\
\vspace*{-.15in}
\begin{list1}
\item[] Using convex approximations to the compressed sensing recovery problem, we reconstructed near-exact versions of images at extremely low compressions, with proofs of correctness. Code \href{https://github.com/alankarkotwal/cs-rank-minimization}{\textcolor{blue} {here}}.
\end{list1}

\vspace*{-0.1in}

{\bf Hidden Markov Model Part-of-Speech Tagging} \hfill \textit{EE638: Estimation and Identification} \\
{\em Guide: \href{https://www.ee.iitb.ac.in/course/~ee638/Navin}{\textcolor{blue}{Prof. Navin Khaneja}}, EE, IITB \hfill Autumn 2015-16} \\
\vspace*{-.15in}
\begin{list1}
\item[] We implemented part-of-speech tagging with support for unknown words. An error rate of around 5\% and capabilities of the system to discern context were observed. Code \href{https://github.com/alankarkotwal/pos-tagging}{\textcolor{blue} {here}}.
\end{list1}

\vspace*{-0.1in}

{\bf Laparoscopic Image Dehazing With Dark Channel Prior} \hfill \textit{CS736: Medical Image Processing} \\
{\em Guide: \href{https://www.cse.iitb.ac.in/~suyash}{\textcolor{blue}{Prof. Suyash Awate}}, CSE, IITB \hfill Spring 2014-15} \\
\vspace*{-.15in}
\begin{list1}
\item[] We applied the Dark Channel Prior method for landscape image dehazing to surgical smoke--affected laparoscopic images, accelerated it in time and got good results. Code \href{https://github.com/riddhishb/Laproscopic-Image-Dehazing}{\textcolor{blue} {here}}.
\end{list1}

\vspace*{-0.1in}

{\bf Stereo Odometry Via Point Cloud Registration} \hfill \textit{CS763: Computer Vision} \\
{\em Guide: \href{https://www.cse.iitb.ac.in/~ajitvr}{\textcolor{blue}{Prof. Ajit Rajwade}}, CSE, IITB \hfill Spring 2014-15} \\
\vspace*{-.15in}
\begin{list1}
\item[] Maximizing kernel density correlation with gradient-ascent and coherent point drift, we registered pointclouds and observed good convergence behavior for small transformations. Code \href{https://github.com/alankarkotwal/stereo-vo}{\textcolor{blue} {here}}.
\end{list1}

\vspace*{-0.1in}

{\bf Gravitational Lens Separation With PCA} \hfill \textit{CS663: Digital Image Processing} \\
{\em Guide: \href{https://www.cse.iitb.ac.in/~suyash}{\textcolor{blue}{Prof. Suyash Awate}} \& \href{https://www.cse.iitb.ac.in/~ajitvr}{\textcolor{blue}{Prof. Ajit Rajwade}}, CSE, IITB \hfill Autumn 2014-15} \\
\vspace*{-.15in}
\begin{list1}
\item[] Gravitationally lensed images of galaxies have rare arc-like artifacts that can be used to calculate the mass of the lens. We used Anscombe denoising followed by PCA to build a basis for galaxy images and used the top few eigengalaxies to subtract sources and detect arcs. Code \href{https://github.com/alankarkotwal/pca-lens-finder}{\textcolor{blue} {here}}.
\end{list1}

\vspace*{-0.1in}

{\bf Processor Design} \hfill \textit{EE309: Microprocessors} \\
{\em Guide: \href{https://www.ee.iitb.ac.in/~viren/}{\textcolor{blue}{Prof. Virendra Singh}}, EE, IITB \hfill Autumn 2014-15} \\
\vspace*{-.15in}
\begin{list1}
\item[] We designed, simulated and implemented (on a DE0-Nano board from Terasic) a \href{https://github.com/alankarkotwal/lc-3b-processor}{\textcolor{blue} {multi-cycle RISC processor}} following the LC-3b ISA. Following this, we designed and simulated a \href{https://github.com/alankarkotwal/lca-processor}{\textcolor{blue} {pipelined RISC processor}} using the Little Computer Architecture.
\end{list1}

\section{\sc Astrophysics \\Projects}
{\bf Detecting Short $\gamma$-Ray Bursts in Astrosat CZTI Data} \hfill \textit{PH426: Astrophysics} \\
{\em Guide: \href{https://sites.google.com/site/vikramrentalahome/}{\textcolor{blue}{Prof. Vikram Rentala}}, \textit{PH, IITB} and \href{http://web.tifr.res.in/~arrao/}{\textcolor{blue}{Prof. A. R. Rao}}, \href{http://www.tifr.res.in/}{\textcolor{blue} {TIFR, Mumbai}} \hfill Spring 2015-16} \\
\vspace*{-.15in}
\begin{list1}
\item[] We did a literature survey on $\gamma$-ray bursts, including open problems in the field. We tackle detecting short $\gamma$-ray bursts from data acquired by the CZTI X-Ray Imager on-board Astrosat.
\end{list1}

\vspace*{-0.1in}

{\bf Variability Analysis for Globular Cluster NGC2419} \hfill \textit{\href{http://nius.hbcse.tifr.res.in/}{\textcolor{blue} {NIUS, Astronomy}}} \\
{\em Guide: \href{http://http://manuu.ac.in/deptphysc_faclty.php/}{\textcolor{blue}{Prof. Priya Hasan}}, \href{http://manuu.ac.in/}{\textcolor{blue} {MANUU, Hyderabad}} \hfill December 2015} \\
\vspace*{-.15in}
\begin{list1}
\item[] We analyzed raw data for the globular cluster NGC2419 taken at the \href{http://www.iiap.res.in/iao/cycle.html}{\textcolor{blue} {HCT}}, post-processed it to correct for detector bias and flat-fielding, inverted the effect of atmospheric mass and extracted the variation of magnitudes of stars in the cluster on the scale of a day. Code \href{https://github.com/alankarkotwal/ngc2419-variables}{\textcolor{blue} {here}}.
\end{list1}

\vspace*{-0.1in}

{\bf An X-Ray Study of Black Hole Candidate X Norma X-1} \hfill \textit{\href{http://nius.hbcse.tifr.res.in/}{\textcolor{blue} {NIUS, Astronomy}}} \\
{\em Guide: \href{http://cbs.ac.in/people/visiting-scientists/manojendu-choudhury}{\textcolor{blue}{Prof. Manojendu Choudhury}}, \href{http://cbs.ac.in/}{\textcolor{blue} {Center for Basic Sciences}} \hfill December 2013} \\
\vspace*{-.15in}
\begin{list1}
\item[] We analyzed spectral data from the RXTE for the low-mass X-Ray Binary 4U 1630-47, for an outburst in the source. We extracted 3-30 keV spectra and fit them with a model accounting for disk blackbody radiation, non-thermal power-law radiation, and interstellar extinction. We obtained best fit values of system parameters like internal radius and temperature. Report \href{https://alankarkotwal.github.io/4U_1630-47_Report.pdf}{\textcolor{blue} {here}}.
\end{list1}

\vspace*{-0.1in}

{\bf Estimation of Photometric Redshifts Using Machine Learning} \hfill \textit{\href{http://nius.hbcse.tifr.res.in/}{\textcolor{blue} {NIUS, Astronomy}}} \\
{\em Guide: \href{http://www.iucaa.ernet.in/~nspp/}{\textcolor{blue}{Prof. Ninan Sajeeth Philip}}, \href{http://www.iucaa.ernet.in/}{\textcolor{blue} {IUCAA}}, Pune \hfill December 2012} \\
\vspace*{-.15in}
\begin{list1}
\item[] Here, we trained a neural network for photometric redshifts, given data for sources whose spectra and redshifts have been measured. We predicted spectra for these objects viewed at various other values of redshifts. Using this expanded dataset, we achieved good predictions for test data.
\end{list1}

\section{\sc Achievements \\ and Awards}
\begin{list2}
\item[\strut\hspace{0.5cm}\textbf{Olympiads and Competitive Exams}]
\item Represented India at the \href{http://www.ioaa2012.ufrj.br/}{\textcolor{blue} {$6^\text{th}$ International Olympiad on Astronomy and Astrophysics}}, Brazil, 2012. Won a Gold Medal with International Rank 4 and a special prize for Best Data Analysis
\item Represented India at the \href{http://www.ieso2011.unimore.it/}{\textcolor{blue} {$5^\text{th}$ International Earth Sciences Olympiad}}, Italy, 2011. Won a Bronze Medal and prizes for best performance in the Hydrosphere section and the team presentation
\item  Secured All India Rank (AIR) $105$ in \href{https://en.wikipedia.org/wiki/Indian_Institute_of_Technology_Joint_Entrance_Examination}{\textcolor{blue}{IIT-JEE}} amongst $1.1$ million candidates
\vspace{0.05in}
\item[\strut\hspace{0.5cm}\textbf{Scholarships}]
\item Awarded \href{http://www.kvpy.iisc.ernet.in/main/index.htm}{\textcolor{blue}{KVPY Scholarship}} $2011$ by Dept. of Science and Technology, Govt. of India
\item Awarded \href{http://www.ncert.nic.in/programmes/talent_exam/index_talent.html}{\textcolor{blue}{NTSE Scholarship}} $2008$ by NCERT, Govt. of India
\item[\strut\hspace{0.5cm}\textbf{Competitions}]
\item Secured IIT Bombay the second position by putting on board 72 Messier objects including the entire Virgo cluster of galaxies in the \href[page=2]{https://www.iitbombay.org/iitb_dean_acr/february-newsletter-2014/2nd\%20Annual\%20Inter.pdf}{\textcolor{blue} {Inter-IIT Messier Marathon, 2014}}
%\item Won the Astronomy Quiz conducted by the Astronomy Club, IIT Bombay, 2012
\end{list2}

\section{\sc Key Talks \\ and Seminars}
{\bf Coded Source Separation for Compressed Video Recovery} \hfill {\em Dual Degree Thesis Talk} \\
{\em \href{http://www.ee.iitb.ac.in/}{\textcolor{blue}{Department of Electrical Engineering}}, \href{http://www.iitb.ac.in/}{\textcolor{blue}{Indian Institute of Technology Bombay}} \hfill May 2016} \\
\vspace*{-.15in}
\begin{list1}
\item[] Here, I presented results from the first stage of my dual degree thesis. Presentation \href{http://alankarkotwal.github.io/sre.pptx}{\textcolor{blue} {here}}.
\end{list1}

\vspace*{-0.1in}

{\bf Template-Based Stereo Odometry} \hfill {\em Invited Talk} \\
{\em \href{http://theairlab.org/}{\textcolor{blue}{The AIR Lab}}, \href{http://www.cmu.edu/}{\textcolor{blue}{Carnegie Mellon University}} \hfill July 2015} \\
\vspace*{-.15in}
\begin{list1}
\item[] Here, I presented results from my 2015 summer internship to my group at Carnegie Mellon University. The talk included a detailed description of the method used, comparisons of the results with ground-truth and stress-tests on the method. Presentation \href{http://alankarkotwal.github.io/intern_presentation.pptx}{\textcolor{blue} {here}}.
\end{list1}

\vspace*{-0.1in}

{\bf \href{http://www.stab-iitb.org/krittika/the-cosmic-ladder-distance}{\textcolor{blue} {The Cosmic Distance Ladder}}} \hfill {\em Invited Talk} \\
{\em \href{http://www.stab-iitb.org/krittika/}{\textcolor{blue} {Krittika -- The Astronomy Club, IIT Bombay}} \hfill September 2014, February 2016} \\
\vspace*{-.15in}
\begin{list1}
\item[] This open-to-all popular talk is climbs the Cosmic Distance Ladder, a sequence of steps, each building on the previous step's results, for calculating distances in the universe. We begin with solar system distances, and end at enormous distances where the only option is using indirect methods like photometric redshifts. Presentation \href{http://alankarkotwal.github.io/CosmicDistanceLadder.pptx}{\textcolor{blue} {here}}.
\end{list1}

\section{\sc Mentoring Experience}
\textbf{Teaching Assistant for IITB Courses}
\begin{list1}
\item[]CS663: Digital Image Processing \hspace{0.5cm} \href{https://www.cse.iitb.ac.in/~suyash}{\textcolor{blue}{Prof. S. Awate}} \& \href{https://www.cse.iitb.ac.in/~ajitvr}{\textcolor{blue}{Prof. A. Rajwade}} \hfill{\textit{Autumn 2015-16}}
\item[]CS736: Medical Image Processing \hspace{2cm} \href{https://www.cse.iitb.ac.in/~suyash}{\textcolor{blue}{Prof. S. Awate}}\hfill{\textit{Spring 2015-16}}
\end{list1}

\vspace*{-0.1in}

\textbf{Resource Person, Indian Astronomy Olympiad Programme} \hfill \textit{May 2013, May 2014} \\
\vspace*{-.15in}
\begin{list1}
\item[] Selected twice as a resource person for the Indian Astronomy Olympiad Camp, for their selection to the international Astronomy Olympiads. Involved in mentoring students ranging from the $9^\text{th}$ to the $12^\text{th}$ grades in Astronomy, and in setting up challenging questions and evaluating students.
\end{list1}

\vspace*{-0.1in}

\newpage
\textbf{Technical Mentor}\hfill \textit{April 2013 -- March 2014} \\
\vspace*{-.15in}
\begin{list1}
\item[] Mentored $1^\text{st}$ year students for Robotics Competitions and contributed towards generation of ideas and debugging for Institute Technical Summer Projects. Member of the Electronics club.
\end{list1}

\section{\sc Key \\Coursework} 
\begin{list1}
\item[\strut\hspace{0.5cm}\hypertarget{crselst}{\textbf{Computer Sciences and Engineering}}]
\item[]\textit{Computer Vision, Medical Image Processing, Digital Image Processing, Machine Learning, Convex Optimization, Graphics, Networks, Data Structures, Algorithms, Discrete Mathematics}
\item[\strut\hspace{0.5cm}\textbf{Electrical Engineering}]
\vspace{0.05in}
\item[]\textit{Estimation and Identification, Adaptive Signal Processing, Digital Signal Processing, Speech Processing, Matrix Computations, Information Theory, Advanced Probability and Random Processes, Communication Networks and Systems, Microprocessors, Signals and Systems, Digital and Analog Systems, Electronic Devices and Circuits, Network Theory}
\item[\strut\hspace{0.5cm}\textbf{Physics and Mathematics}]
\vspace{0.05in}
\item[]\textit{Astrophysics, The General Theory of Relativity, Electromagnetic Waves, Electricity \& Magnetism, Classical Mechanics, Differential Equations, Linear Algebra, Complex Analysis, Calculus}
\end{list1}

\section{\sc Technical \\Skills} 
\begin{tabular}{@{}p{1.3in}p{4.3in}}
\textbf{Programming} & C/C++, Python, Bash, Matlab, Verilog, SQL, HTML/CSS, PHP, \LaTeX \\  
\vspace*{-0.06in}
\textbf{Software Packages} & 
\vspace*{-0.06in}
ROS/Gazebo, OpenCV, The Point Cloud Library, SPICE Circuit Simulation, EAGLE PCB Design, SolidWorks, AutoCAD, LabView\\ 
\vspace*{-0.06in}
\textbf{Science Software} &
\vspace*{-0.06in}
Python packages: NumPy, SciPy and Matplotlib, GNUPlot, Scikit-learn, Astropy, SDSS tools \\
\vspace*{-0.06in}
\textbf{Hardware} &
\vspace*{-0.06in}
\textit{Microprocessors:} 8051, 8085, AVR and PIC, CPLDs and FPGAs, \textit{Embedded Platforms:} Arduino, Raspberry Pi, NVIDIA Jetson TK1 \textit{standard digital logic families} \\     
\end{tabular}

\section{\sc Languages}
English, Hindi (working knowledge), Marathi (first language), Sanskrit (elementary knowledge)

\section{\sc Other \\Interests}
\lettrine[lines=2]{O}{ther} than my academic interests, I like biking, long walks, trekking, climbing whatever can be climbed, swimming, socializing, cooking good food and eating it. I especially enjoy classic rock music and people who enjoy my interests. I enjoy design and like making things look and feel good. And finally, I love Origami, the art of paper folding, and building complex, realistic models with Lego blocks.

\section{\sc References} 
\begin{tabular}{@{}p{3in}p{3in}}
\textbf{Prof. Suyash Awate}, CSE & \textbf{Prof. Ajit Rajwade}, CSE \\ 
Indian Institute of Technology, Bombay & Indian Institute of Technology, Bombay \\
\href{mailto:suyash@cse.iitb.ac.in}{\textcolor{blue}{E--Mail}} $|$ \href{https://www.cse.iitb.ac.in/~suyash}{\textcolor{blue}{Webpage}} & \href{mailto:ajitvr@cse.iitb.ac.in}{\textcolor{blue}{E--Mail}} $|$ \href{https://www.cse.iitb.ac.in/~ajitvr}{\textcolor{blue}{Webpage}} \\
\vspace{1pt} & \vspace{1pt} \\
\textbf{Dr. Sebastian Scherer}, Robotics Institute & \textbf{Ashutosh Richhariya}, Ophthalmic Biophysics \\ 
Carnegie Mellon University & L. V. Prasad Eye Institute \\
\href{mailto:basti@andrew.cmu.edu}{\textcolor{blue}{E--Mail}} $|$ \href{http://www.ri.cmu.edu/person.html?person_id=1397}{\textcolor{blue}{Webpage}} & \href{mailto:ashutosh@lvpei.org}{\textcolor{blue}{E--Mail}} $|$ \href{http://www.lvpei.org/our-team/our-team-ashutosh.php}{\textcolor{blue}{Webpage}} \\
\vspace{1pt} & \vspace{1pt} \\
\textbf{Prof. Mayank Vahia}, Astrophysics & \textbf{Dr. Aniket Sule}, Astronomy \\ 
Tata Institute of Fundamental Research & Homi Bhabha Center for Science Education \\
\href{mailto:vahia@tifr.res.in}{\textcolor{blue}{E--Mail}} $|$ \href{http://www.tifr.res.in/~vahia/}{\textcolor{blue}{Webpage}} & \href{mailto:anikets@hbcse.tifr.res.in}{\textcolor{blue}{E--Mail}} $|$ \href{http://www.hbcse.tifr.res.in/people/academic/aniket-sule}{\textcolor{blue}{Webpage}} \\
\vspace{1pt} & \vspace{1pt} \\
\textbf{Prof. Rajbabu Velmurugan}, EE & \textbf{Dr. Manojendu Choudhury}, Astrophysics \\ 
Indian Institute of Technology, Bombay & Center for Excellence in Basic Sciences \\
\href{mailto:rajbabu@ee.iitb.ac.in}{\textcolor{blue}{E--Mail}} $|$ \href{https://www.ee.iitb.ac.in/web/faculty/homepage/rajbabu}{\textcolor{blue}{Webpage}} & \href{mailto:manojendu@cbs.ac.in}{\textcolor{blue}{E--Mail}} $|$ \href{http://www.cbs.ac.in/people/physics-faculty/manojendu-choudhury}{\textcolor{blue}{Webpage}} \\
\end{tabular}
\end{resume}
\end{document}